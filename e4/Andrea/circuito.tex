\section*{Materiale}

Il materiale utilizzato per questa esperienza di laboratorio è il sguente:

\begin{itemize} \itemsep2pt \parskip0pt \parsep0pt
    \item{Breadboard, cavetti e cavi a banana per le connessioni;}
    \item{Amplificatore operazionale UA741 a 8 pin;}
    \item{Comparatore LM311 a 8 pin;}
    \item{Un fotodiodo , un led e capacità varie;}
    \item{Resistenze varie ed una resistenza variabile tra $0$ e $10$ \SI{}{\kilo\ohm};}
    \item{Generatore di funzioni d'onda: Agilent 33120A;}
    \item{Multimetro: Agilent Technologies 34410A;}
    \item{Oscilloscopio: Agilent DSO-X 2002A;}
\end{itemize}

Facciamo presente che sui valori di resistenza riportati in tutto l'elaborato non abbiamo riportato la loro incertezza per motivi di chiarezza e leggibilità dello scritto. Tuttavia abbiamo assunto un errore del $5\%$ sul valore nominale delle resistenze.

\section*{Circuiti}

%\begin{figure}[h]
%        \centering
%        \begin{subfigure}[b]{0.45\textwidth}
%        	\def\svgwidth{\textwidth}
%                \subimport{Figure/}{slew_circ.pdf_tex}
%                \caption{Circuito usato per misurare lo Slew Rate dell'amplificatore operazionale. E' un circuito in configurazione emitter follower, con circuito di retroazione negativo.}
%                \label{fig:slew_rate}
%        \end{subfigure}
%        ~
%        \begin{subfigure}[b]{0.45\textwidth}
%        	\def\svgwidth{\textwidth}
%                \subimport{Figure/}{max_curr_circ.pdf_tex}
%                \caption{Circuito usato per misurare il valore della corrente massima generabile dall'amplificatore. L'amplificatore operazionale è usato in configurazione emitter follower con circuito di retroazione negativo.}
%                \label{fig:current}
%        \end{subfigure}
%        \caption{}
%\end{figure}
