\section*{Conclusione}

Come è ben visibile dai due grafici, rispettivamente in Figura \ref{fig:ecg_normale} e \ref{fig:ecg_corsa}, l'andamentotipico di una cardioide è rispettato. Questo è un risultato rassicurante in quanto, se non avessimo ottenuto tali risultati o era presete un errore nella realizzazione del circuito o il paziente non godeva di uno stato di salute ottimale!
Inoltre possiamo notare come, correttamente, i battiti cardiaci da uno stato di quiete, durante il quale il paziente aveva circa 75 battiti/minuto (Figura \ref{fig:ecg_normale}), si è passati a 150 battiti/minuto in uno stato di sforzo fisico quale uno sprint di corsa (Figura \ref{fig:ecg_corsa}) che sono esattamente il doppio dei battiti a regime normale. 