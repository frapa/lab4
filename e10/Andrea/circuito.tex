\section*{Materiale}

Il materiale utilizzato per questa esperienza di laboratorio è il seguente:

\begin{itemize} \itemsep2pt \parskip0pt \parsep0pt
    \item{Breadboard, cavetti e cavi a banana per le connessioni;}
    \item{Resistenza da \SI{1}{\kilo\ohm}, un trimmer da \SI{1}{\kilo\ohm} e un diodo LED;}
    \item{Circuito integrato SN74LS00N a 14 pin;}
    \item{Circuito integrato SN74LS04N a 14 pin;}
	\item{Circuito integrato SN74LS05N a 14 pin;}
	\item{Porta buffer TTL 3-state 74LS125A;}
	\item{Una schedina per la visualizzazione dello stato di uscite digitali a LED;}
    \item{Alimentatore di tensione continua;}
    \item{Generatore di funzioni d'onda: Agilent 33120A;}
    \item{Multimetro: Agilent Technologies 34410A;}
    \item{Oscilloscopio: Agilent DSO-X 2002A;}
\end{itemize}

Facciamo presente che sui valori di resistenza riportati in tutto l'elaborato non abbiamo riportato la loro incertezza per motivi di chiarezza e leggibilità dello scritto. Tuttavia abbiamo assunto un errore del $5\%$ sul valore nominale delle resistenze.

%\section*{Circuito}