\section*{Dati e risultati}

\subsection*{Sorgente di corrente costante}

Vogliamo verificare che il primo circuito realizzato (Figura \ref{fig:sorgente_corrente}) sia una sorgente di corrente costante. Ovvero, facendo riferimento alla Figura \ref{fig:sorgente_corrente}, abbiamo realizzato un circuito che mantiene l'intensità di corrente $I_0$ del ramo di retroazione negativa ad un valore costante di $\SI{1}{\milli\ampere}$, per qualunque valore assunto dalla resisenza variabile $R_v$.

A tal fine abbiamo alimentato l'amplificatore operazionale con una tensione positiva $V\ped{cc}^+\,=\,\SI{+15}{\volt}$ e una negativa $V\ped{cc}^-\,=\,\SI{-15}{\volt}$ e abbiamo fornito una tensione in ingresso di $V\ped{in}\,=\,\SI{+3.9}{\volt}$.

Per verificare il corretto funzionamento del circuito abbiamo misurato l'intensità di corrente $I_0$ per vari valori di $R_v$. I risultati ottenuti sono riportati in Tabella \ref{tab:corr_res}.


\begin{table}[H]
    \centering
    \small
    \begin{tabular}{l | c}
        \toprule
		Corrente $I_0 \, [\si{\milli\ampere}]$ & $R_v \; [\si{\kilo\ohm}]$ \\
        \midrule
		$ 1.009 \pm 0.0005 $ & $ 10 $ \\
		$ 1.009 \pm 0.0005 $ & $ 9 $ \\
		$ 1.009 \pm 0.0005 $ & $ 8 $ \\
		$ 1.009 \pm 0.0005 $ & $ 7 $ \\
		$ 1.009 \pm 0.0005 $ & $ 6 $ \\
		$ 1.009 \pm 0.0005 $ & $ 5 $ \\
		$ 1.009 \pm 0.0005 $ & $ 4 $ \\
		$ 1.009 \pm 0.0005 $ & $ 3 $ \\
		$ 1.009 \pm 0.0005 $ & $ 2 $ \\
		$ 1.009 \pm 0.0005 $ & $ 1 $ \\
        \bottomrule
    \end{tabular}
    \caption{In questa tabella sono riportati i valori dell'intensità di corrente $I_0$, passante per il ramo di retroazione negativa del circuito in Figura \ref{fig:sorgente_corrente}, al variare della resisenza $R_v$.}
    \label{tab:corr_res}
\end{table}

Facciamo notare che gli errori sui valori di resistenza  

\subsection*{Sommatore pesato di segnali in ingresso}

In questa sezione vogliamo montare e studiare il corretto funzionamento di un circuito che compie una sommatoria pesata dei segnali in ingresso. Questo circuito, riportato in Figura \ref{fig:sommatore}, è stato realizzato nel seguente modo: abbiamo alimentato l'amplificatore operazionale con una tensione positiva $V\ped{cc}^+\,=\,\SI{+15}{\volt}$ e una negativa $V\ped{cc}^-\,=\,\SI{-15}{\volt}$, abbiamo fornito in ingresso due forme d'onda con la stessa frequenza $\nu\,=\,\SI{1}{\kilo\hertz}$, ma con valori di tensione picco-picco differenti. Riferendoci sempre alla Figura \ref{fig:sommatore} i valori di resistenza che abbiamo scelto sono stati i seguenti: $R_1\,=\,\SI{50}{\kilo\ohm}$ e $R_2\,=\,\SI{100}{\kilo\ohm}$ ed infine $R_f\,=\,\SI{3.9}{\kilo\ohm}$.

Grazie all'utilizzo dell'osciloscopio siamo stati in grado di appurarne il corretto funzionamento osservando l'andamento dei due segnali in ingresso ($V\ped{in1}\,e\,V\ped{in2} $) e del seganle in uscita ($V\ped{out}$), che come ci si poteva aspettare e risultato essere la somma dei due segnali in ingresso invertiti.

Inoltre, sfruttando le due regole fondamentali per un amplificatore operazionale, ovvero:
\begin{itemize}
	\item{Prima: la corrente che passa tra l'ingresso positivo e quelo negativo deve essere nulla;}
	\item{Seconda: la differenza di potenziale tra lasorgente e uno dei due ingressi deve essere sempre uguale alla differenza di potenziale tra l'uscita e l'ingresso, in modo tale che, grazie alle leggi di Kirchhoff, la corrente passante per i due rami sia di pari intensità;}
\end{itemize}










