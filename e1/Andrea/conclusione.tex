\section*{Conclusione}

\subsection*{Sorgente di corrente costante}

Come si può osservare dai dati riportati in Tabella \ref{tab:corr_res} l'intensità di corrente $I_0$ non varia al variare della resistenza variabile $R_v$, e rimane costante, nel nostro caso ad un valore di $I_0\,=\,\SI{1.009}{\milli\ampere}$. Pertanto possiamo concludere che, come ci aspettavamo, il circuito realizzato si comporta come una sorgente di corrente costante.

\subsection*{Sommatore pesato di segnali in ingresso}

Per quanto rigurda il circuito sommatore, schematizzato in Figura \ref{fig:sommatore}, possiamo dire che, sfruttando l'oscilloscopio, siamo riusciti a verficarne il corretto comportamento.
Purtroppo negli screenshot delle forme d'onda acquisite dall'oscilloscopio che abbiamo preso, illustranti l'andamento dei tre segnali ($V\ped{in1},\,\,V\ped{in2}\,\,e\,\,V\ped{out}$), non è riportata la persistenza di ($V\ped{out}$) e quindi l'immagine risulta inutile in quanto non fornisce indicazioni sull'output dell circuito. Per questo motivo non possiamo fornire un'immagine illustrativa di quello che è il reale comportamento del circuito realizzato.