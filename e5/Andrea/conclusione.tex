\section*{Conclusione}

Per quanto riguarda il raddrizzatore di precisione a semionda possimo dire che tutto ha funzionato come da previsioni teoriche, fatta eccezione per il ritardo nella risposta di $V\ped{out}$ in salita già discusso ampliamente nella sezone precedente. A tal fine abbiamo realizzato un raddrizzatore di precisione a semionda ottimizzato, che come è possibile osservare dal grafico in Figura \ref{fig:radd_ott_plot}, ha migliorato notevolmente il risultato ottenuto in precedenza (Figura \ref{fig:radd_plot1}).
Per quanto riguarda l'amlificatore differenziale possimo solamente dire che ha lavorato correttamente e fintanto che il rumore in ingresso al circuito non è risultato essere molto maggiore del segnale che si volevamo amplificare e i suoi effetti sono stati completamente eliminati.
Per quanto riguarda l'amplificatore per strumentazione possiamo dire che tutto ha funzionato al meglio. Per l'analisi del circuito e delle sue caratteristiche rimandiamo alla sezione precedente dove è spegato tutto in dettaglio.
Infine per quanto rigurda la misura di temperatura mediante l'utilizzo di un trasduttore di temperatura Pt100 possimo dire che, come già esposto, questa misura conviene effettuarla a 4 fili dal momento che, in questa configurazione, è possibile rendere nullo il contributo di impedenza dovuto ai cavi di collegamento. 

