\documentclass[11pt, twoside, a4paper]{article}
\usepackage[italian]{babel}
\usepackage[utf8]{inputenc}
\usepackage{amsmath}
\usepackage[cm]{fullpage}
\usepackage{graphicx}
\usepackage{booktabs}
\usepackage{wrapfig}
\usepackage{multirow}
\usepackage{sidecap}
\usepackage{siunitx}
\usepackage{array}
\usepackage[font=small]{caption}
%\usepackage[bookmarks, hidelinks]{hyperref}
\usepackage{float}
\usepackage{wrapfig}
\usepackage{titlesec}
\usepackage{subcaption}
\usepackage{import}
\usepackage{mdwlist}

\titlespacing\section{0pt}{10pt plus 4pt minus 4pt}{8pt plus 2pt minus 4pt}
\titlespacing\subsection{0pt}{10pt plus 4pt minus 4pt}{8pt plus 2pt minus 4pt}
\titlespacing\subsubsection{0pt}{10pt plus 4pt minus 4pt}{8pt plus 2pt minus 4pt}

\begin{document}
  \begin{center}
  
    {\huge Esperienza 3: Raddrizzatori di precisione $\&$ Amplificatori operazionali}
    
    \vspace{0.1cm}
    
    {Francesco Pasa, Andrea Miani - Gruppo A05} \\
    {14 ottobre 2014}
    
    \vspace{-0.1cm}
    
  \end{center}
  
  \section*{Obbiettivo}

In questa sessione di laboratorio vogliamo come in primis andare a studiare un raddrizzatore di tensione a semionda. Partiremo con il circuito più semplice per poi provarne una configurazione più efficace.
Come secondo obbiettivo ci siamo proposti di andare a verificare come sia possibile con un amplificatore differenziale sopprimere gli effetti di rumore presenti ad un segnale in ingresso.
Successivamente abbiamo deciso verificare il funzionamento di un amplificatore per srumentazione.
Infine vogliamo fare delle misure di temoeratura sfruttando un trasduttore di temperatura Pt100.


  \subsection{Materiali e circuiti}

Ci siamo serviti dei seguenti materiali:

\begin{itemize}
    \item{Breadboard, cavi a banana e cavetti da breadboard.}
    \item{Amplificatore operazionale UA741.}
    \item{Resistenze: \SI{100}{\ohm}, \SI{2.2}{\kilo\ohm}, \SI{10}{\kilo\ohm}, \SI{100}{\kilo\ohm}
        e una variabile per aggiustare l'offset dell'amplificatore operazionale.
        Nel nostro caso abbiamo usato una resistenza trimmer con un range operativo da 0
        a \SI{10}{\kilo\ohm}.}
    \item{Condensatore da 180 pF.}
    \item{Alimentatore di corrente continua.}
    \item{Generatore di forme d'onda Agilent 33120A.}
    \item{Multimetro Agilent 34410A.}
    \item{Oscilloscopio Agilent DSO-X 2002A.}
\end{itemize}

Per le misure abbiamo utilizzato i circuiti riportati in figura \ref{fig:circuits3},
che verranno spiegati nei seguenti paragrafi.

\begin{figure*}[t]
        \centering
        \small
        \begin{subfigure}[b]{0.32\textwidth}
            \def\svgwidth{\columnwidth}
            \subimport{figure/}{slew_circ.pdf_tex}
            \caption{Circuito utilizzato per misuraro lo slew rate dell'opamp. Il circuito è
		riportato sul manuale dell'operazionale ed è lo standard per questo tipo di misure.}
            \label{fig:slew_circ3}
        \end{subfigure}
        ~
        \begin{subfigure}[b]{0.32\textwidth}
            \def\svgwidth{\columnwidth}
            \subimport{figure/}{max_curr_circ.pdf_tex}
            \caption{Per la misura della massima corrente erogabile dall'operazionale ci siamo
		serviti del circuito in figura.}
            \label{fig:max_curr_circ3}
        \end{subfigure}
        ~
        \begin{subfigure}[b]{0.32\textwidth}
            \def\svgwidth{\columnwidth}
            \subimport{figure/}{banda_circ.pdf_tex}
            \caption{Circuito di cui abbiamo misurato la banda passante. Il circuito è stato utilizzato
		in due differenti configurazioni: una con guadagno di 20 dB e $R_2 = \SI{10}{\kilo\ohm}$
		e l'altro con un guadagno di 20 dB e $R_2 = \SI{100}{\kilo\ohm}$.}
            \label{fig:banda_circ3}
        \end{subfigure}
        ~
        \begin{subfigure}[b]{0.32\textwidth}
            \def\svgwidth{\columnwidth}
	    \vspace{0.5cm}
            \subimport{figure/}{guad_freq_bassa_circ.pdf_tex}
	    \vspace{0pt}
            \caption{Circuito utilizzato per misurare il guadagno open loop di un amplificatore
		operazionale. Il circuito è pensato per fare misure a basse frequenze, dove
		l'operazionale ha un guadagno molto elevato (circa $10^5$). Quando l'operazionale viene
		usato ad alte frequenze il suo guadagno si riduce notevolmente e il circuito in figura
		diventa inutilizzabile. In questi regimi si utilizza il circuito \ref{eq:gain_high_circ3}.}
            \label{fig:gain_low_circ3}
        \end{subfigure}
        ~
        \begin{subfigure}[b]{0.32\textwidth}
            \def\svgwidth{\columnwidth}
            \subimport{figure/}{guad_freq_alta_circ.pdf_tex}
	    \vspace{0pt}
            \caption{Circuito per la misura del guadagno open loop ad alte frequenze. In circuito in pratica
		serve a fare una misura diretta del guadagno, tuttavia utilizza un ramo di feedback per impedire
		saturazioni non volute (per esempio causate dalla tensione di offset, che non può mai essere
		perfettamente oppure da altre tensioni DC dovute all'alimentazione che vengono amplificate moltissimo
		perché a bassa frequenza).}
            \label{fig:gain_high_circ3}
        \end{subfigure}

        \caption{Circuiti costruiti durante l'esperienza}
        \label{fig:circuits3}
\end{figure*}

  \subsection{Dati e risultati}

\subsubsection{Generatore di corrente costante.}

Il generatore di corrente costante è stato costruito come nello schema in
Figura \ref{fig:generatore}. La scelta della tensione di input V è stata dettata
dal valore della resistenza R a nostra disposizione e dalla corrente che volevamo
generare: 1 mA. Infatti il polo invertente dell'operazionale è un ground virtuale
(cioè $V_A = 0$), quindi la corrente $I_0$, tenuto conto del fatto che il polo
assorbe una corrente trascurabile, vale $V/R$ (1 mA appunto). 

Poiché abbiamo usato una resistenza R con una tolleranza del 5\%,
che assumo come incertezza sul valore della stessa, e che l'incertezza di risoluzione
sulla tensione V è di 0.005 V, il valore atteso della corrente con l'incertezza è
$I_0 = 1 \pm 0.05$ mA.

Abbiamo misurato con il multimetro la corrente $I_0$ al variare del valore della
resistenza $R_v$, per verificare il funzionamento del generatore. La noiosa
Tabella \ref{tab:gen_corr1} mostra che la corrente non varia al variare della resistenza
di carico, proprio come volevamo realizzare. Il circuito si comporta come
una sorgente di corrente costante.

\begin{SCtable}[1][h]
    \centering
    \begin{tabular}{c c}
        \toprule
            $I_0 [\si{\milli\ampere}]$ & $R_v [\si{\kilo\ohm}]$ \\
        \midrule
            $ 1.009 \pm 0.0005 $ & $ 10 $ \\
            $ 1.009 \pm 0.0005 $ & $ 9 $ \\
            $ 1.009 \pm 0.0005 $ & $ 8 $ \\
            $ 1.009 \pm 0.0005 $ & $ 7 $ \\
            $ 1.009 \pm 0.0005 $ & $ 6 $ \\
            $ 1.009 \pm 0.0005 $ & $ 5 $ \\
            $ 1.009 \pm 0.0005 $ & $ 4 $ \\
            $ 1.009 \pm 0.0005 $ & $ 3 $ \\
            $ 1.009 \pm 0.0005 $ & $ 2 $ \\
            $ 1.009 \pm 0.0005 $ & $ 1 $ \\
        \bottomrule
    \end{tabular}
    \caption{La corrente nel circuito \ref{fig:generatore} rimane costante
        al variare della resistenza di carico $R_v$. Le incertezze riportare sul
        valore di corrente sono incertezze di risoluzione del multimetro
        (metà della risoluzione), mentre sui valori di resistenza non sono riportate
        perchè non rilevanti (sono comunque dell'ordine di qualche ohm).}
    \label{tab:gen_corr1}
\end{SCtable}

\subsubsection{Sommatore pesato di tensioni.}

Il sommatore pesato di tensioni che abbiamo realizzato è il circuito \ref{fig:sommatore},
ed è pensato per fornire il seguente output

\begin{equation}
    V\ped{out} = R\left(\frac{V_1}{R_1} + \frac{V_2}{R_2}\right)
    \label{eq:sommatore_pesato1}
\end{equation}

Come nel circuito precedente si ha che $V_A = 0$ (ground virtuale) e che
l'amplificatore operazionale assorbe una quantità di corrente trascurabile,
per cui la corrente di retroazione $I_R$ è data dalla somma di $I_1$ e $I_2$
(per la conservazione della carica).
Le resistenze $R_1$ e $R_2$ trasformano le tensioni in ingresso nelle correnti
$I_1$ e $I_2$, pesandole secondo l'inverso dei valori delle stesse.
Questo implica che $I_R$ dipende dalle tensioni in input pesate,
e quindi anche $V\ped{out} = RI_R$ dipende da esse.

La resistenza $R$ determina il guadagno del circuito. Per esempio per la tensione
$V_1$ il guadagno vale

\begin{equation}
    G = \frac{V\ped{out}}{V_1} = \frac{R}{R_1} = 0.078 \pm 0.006
\end{equation}

dove ho considerato incertezze sulle resistenze pari al 5\%.

Per verificare il corretto funzionamento del circuito abbiamo generato due segnali,
usando il generatore di forme d'onda a nostra disposizione e quello integrato
nell'oscilloscopio, e li abbiamo dati in input al circuito. Poi con l'oscilloscopio
abbiamo verificato che l'output si comportasse secondo la \eqref{eq:sommatore_pesato1}.
Il risultato è stato positivo: abbiamo provato diverse combinazioni di sinusoidi,
onde quadre, rampe e triangoli e in tutti i casi il circuito si è comportato correttamente.

Purtroppo l'oscilloscopio a nostra disposizione non ha 3 canali in ingresso (che sarebbero
stati utili per vedere contemporaneamente i due input e l'output), per cui abbiamo dovuto
usare la funzione di persistenza, che non permette di salvare i dati.

  \subsection{Conclusione}

Anche questa esperienza si è conclusa positivamente, nonostante le difficoltà nel montare il
contatore. Inoltre crediamo che queste conoscenze siano importanti in ambito della fisica sperimentale.

\end{document}
