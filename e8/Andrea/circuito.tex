\section*{Materiale}

Il materiale utilizzato per questa esperienza di laboratorio è il seguente:

\begin{itemize} \itemsep2pt \parskip0pt \parsep0pt
    \item{Breadboard, cavetti e cavi a banana per le connessioni;}
    \item{Amplificatore operazionale UA741 a 8 pin;}
    \item{Resistenze da \SI{1}{\kilo\ohm}, \SI{10}{\kilo\ohm} e una resistenza variabili tra $0$ e $1$ \si{\kilo\ohm};}
    \item{Capacità da \SI{15}{\nano\farad};}
    \item{Alimentatore di tensione continua;}
    \item{Generatore di funzioni d'onda: Agilent 33120A;}
    \item{Multimetro: Agilent Technologies 34410A;}
    \item{Oscilloscopio: Agilent DSO-X 2002A;}
\end{itemize}

Facciamo presente che sui valori di resistenza riportati in tutto l'elaborato non abbiamo riportato la loro incertezza per motivi di chiarezza e leggibilità dello scritto. Tuttavia abbiamo assunto un errore del $5\%$ sul valore nominale delle resistenze.

\section*{Circuito}

\begin{figure}[h]
\centering
    \begin{circuitikz}
        \draw
            (0, 4) node[rground] {}
            to [C, l=$C_2\;10\si{\nano\farad}$] (0, 6)
            to (1, 6)
            to [R, l=$R_3\;10\si{\kilo\ohm}$] (1, 4)
            node[rground] {}
            (1, 6) to [R, l=$R_2\;10\si{\kilo\ohm}$] (1, 8)
            to [C, l=$C_1\;10\si{\nano\farad}$] (6.5, 8)
            to (6.5, 3)
            to [vR, l=$R_V\;1\si{\kilo\ohm}$] (3, 3)
            (5, 5.5) node[op amp, yscale=-1] (opamp) {} 
            (1, 6) to (opamp.+)
            (opamp.-) to (3, 5) to (3, 3)
            to [R, l_=$R_1\;47\si{\ohm}$] (3, 1)
            to [lamp, l=12 V 50 mA] (3, 0)
            node[rground] {}
            (opamp.out) to[short, -o] (7, 5.5)
            node[anchor=west] {$V\ped{out}$}
            (opamp.up) to [short, -o] ++(0, -0.5) node[below] {$-V\ped{cc}$}
            (opamp.down) to [short, -o] ++(0, .5) node[above] {$+V\ped{cc}$}
        ;
    \end{circuitikz}
    \caption{Oscillatore a ponte di Wien.}
    \label{fig:oscillatore}
\end{figure}