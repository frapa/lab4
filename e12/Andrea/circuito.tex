\section*{Materiale}

Il materiale utilizzato per questa esperienza di laboratorio è il seguente:

\begin{itemize} \itemsep2pt \parskip0pt \parsep0pt
    \item{Breadboard, cavetti, cavi a banana per le connessioni.}
    \item{Condensatori da \SI{0.47}{\micro\farad}, \SI{10}{\nano\farad},  resistenze da \SI{1}{\kilo\ohm}, \SI{2}{\kilo\ohm}, \SI{2.2}{\kilo\ohm} e \SI{10}{\kilo\ohm};}
    \item{Circuito integrato SN74LS109 a 16 pin;}
    \item{Circuito integrato SN74LS191 a 16 pin;}
    \item{Circuito integrato DAC08 a 16 pin;}
	\item{Una schedina a LED;}
    \item{Alimentatore di tensione continua;}
    \item{Generatore di funzioni d'onda: Agilent 33120A;}
    \item{Multimetro: Agilent Technologies 34410A;}
    \item{Oscilloscopio: Agilent DSO-X 2002A;}
\end{itemize}

Facciamo presente che sui valori di resistenza riportati in tutto l'elaborato non abbiamo riportato la loro incertezza per motivi di chiarezza e leggibilità dello scritto. Tuttavia abbiamo assunto un errore del $5\%$ sul valore nominale delle resistenze.