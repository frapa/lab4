\section*{Conclusione}

Quello che si può dedurre, da quanto visto finora, è che al fine di poter sfruttare un'amplificatore operazionale reale al meglio delle proprie possibilità bisogna prendere alcuni provvedimenti. Il più importante che abbiamo analizzato è stato l'azzeramento della tensione di offset ($V\ped{os}$), che ci permette di utilizzare il transistor con più facilità, in quanto non vi è più una differenza di potenziale tra i due terminali del nostro amplificatore e quindi il suo comportamento risulta essere più simmetrico. Questo fatto è illustrato in Figura \ref{fig:plot_Vd}.
Inoltre, dal momento che abbiamo misurato il valore della corrente di polarizzazione del ramo non invertente, siamo ora in grado di ricavare la tensione di offset, dalla relazione (\ref{eq:offset1}), senza prendere come trascurabile il contributo dato dal termine $I_{p}^-R_2$. Pertanto otteniamo:

\begin{equation}
	V\ped{os}\,=\,(V\ped{out}+I_{p}^-R_2)\left(\frac{R_1}{R_1+R_2}\right)\,=\,(\pm)\SI{}{\milli\volt}
\end{equation}

e possiamo osservare che grazie a questa correzione la nuova tensione di offst, appena ottenuta risulta compatibile entro le incertezze di misura con quella ricavata dalla misura diretta, che ricordiamo vale $V\ped{os}\,=\,(1.09\pm0.01)\SI{}{\milli\volt}$.

Per concludere possiamo calcolare la corrente di offset $I\ped{os}$. Tale corrente non è altro che la differenza tra le due correnti di ingresso ($I_{p}^+$ e $I_{p}^-$) quando la tensione d'uscita ($V\ped{out}$) è nulla e l'amplificatore a riposo.
Quindi in formule otteniamo che:

\begin{equation}
	I\ped{os}\,=\,|I_{p}^+|-|I_{p}^-|\,=\,(\pm)\SI{}{\nano\ampere}
\end{equation}