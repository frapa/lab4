\section*{Conclusione}

Quello che si può dedurre, da quanto visto finora, è che al fine di poter sfruttare un'amplificatore operazionale reale al meglio delle proprie possibilità bisogna prendere alcuni provvedimenti. Il più importante che abbiamo analizzato è stato l'azzeramento della tensione di offset ($V\ped{os}$), che ci permette di utilizzare il transistor con più facilità, in quanto non vi è più una differenza di potenziale tra i due terminali del nostro amplificatore e quindi il suo comportamento risulta essere più simmetrico. Questo fatto è illustrato in Figura \ref{fig:plot_Vd}.

Abbiamo inoltre verificato che i valori delle correnti di polarizzazione fossero compatibili con quelle cartteristiche di un amplificatore operazionale $UA741$, che normalmente sono comprese tra i 30 e i 70 $\SI{}{\nano\ampere}$.
Per concludere possiamo calcolare la corrente di offset $I\ped{os}$. Tale corrente non è altro che la differenza tra le due correnti di ingresso ($I_{p}^+$ e $I_{p}^-$) quando la tensione d'uscita ($V\ped{out}$) è nulla e l'amplificatore a riposo.
Quindi in formule otteniamo che:

\begin{equation}
	I\ped{os}\,=\,|I_{p}^+|-|I_{p}^-|\,=\,(1.3\pm3.4)\SI{}{\nano\ampere}
\end{equation}