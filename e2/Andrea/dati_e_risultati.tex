\section*{Dati e risultati}

\subsection*{Amplificatore operazionale open loop}

In questo paragrafo voliamo descrivere il comportamento del nostro amplificatore operazionale nella configurazione open loop. Ovvero, come illustrato in Figura \ref{fig:open_loop}, andiamo ad alimentare il nostro OPAMP con una $V\ped{cc}^+\,=\,\SI{+15}{\volt}$ e $V\ped{cc}^-\,=\,\SI{-15}{\volt}$. Colleghiamo i due ingressi, invertente ($V^-$) e non invertente ($V^+$) al comune in modo da non avere alcun segnale in ingresso al nostro amplificatore operazionale. Inoltre non inseriamo alcun circuito di retroazione. Infine colleghiamo l'output all'oscilloscopio, in modo da ottenere il valore della tensione di output ($V\ped{out}$).
A questo punto possiamo notare che: se l'amplificatore operazionale fose ideale la $V\ped{out}$ risulterebbe nulla, in quanto valgono le seguenti relazioni:

\begin{equation}
	V_d\,=\,V^+\,-\,V^- \qquad V\ped{out}\,=\,G\,V_d
\end{equation}

dove con $V_d$ si intende la differenza di potenziale tra i due ingressi dell'OPAMP e con $G$ indichiamo in guadagno del nostro amplificatore operazionale. Pertanto dal momento che i due ingressi si trovano entrambi alla stessa differenza di potenziale la loro differenza risulterebbe nulla e pertanto otterremmo $V\ped{out}\,=\,\SI{0}{\volt}$.
Al contrario per l'amplificatore operazionale reale abbiamo ottenuto che, se $V\ped{out}\,=\,\SI{0}{\volt}$, la differenza di potenziale tra i due ingressi non è 0 ma risulta essere:

\begin{equation}
	V_d\,=\,(-12.80\,\pm\,0.01) \SI{}{\volt}
\end{equation}

Questo valore ci suggerisce che il nostro amplificatore operazionale, già a valori di tensione in ingresso nulli è in saturazione negativa. Per verificare quanto appena detto abbiamo deciso di collegare $V^-$ al comune e $V^+$ a una tensione di $\SI{-15}{\volt}$, in modo che l'OPAMP, fosse sicuramente in saturazione negativa. Quello che abbiamo osservato è che il valore di $V\ped{out}$ raggiunto è di $(-12.94\,\pm\,0.01)\,\SI{}{\volt}$ che è molto vicino al valore di $V_d$ ottenuto. Quindi possiamo dire che in entrambi i casi il nostro amplificatore operazionale si trova in saturazione negativa.

\subsection*{Misura della tensione di offset}

Ricordiamo che la tensione di offset $V\ped{os}$ è quella differenza di potenziale che deve essere applicata ad un amplificatore operazionale per ottenere in uscita una tensione nulla.
Quindi al fine di misurare la tensione di offset del nostro amplificatore operazionale abbiamo montato il circuito riportato in Figura \ref{fig:offset} e lo abbiamo dimensionato in modo da determinare i valori più intelligenti di resistenza atti ad ottenere una buona stima di $V\ped{os}$.
Come per il circuito precedente abbiamo che $V\ped{cc}^+\,=\,\SI{+15}{\volt}$ e $V\ped{cc}^-\,=\,\SI{-15}{\volt}$.
In questo caso, sfruttando le regole per gli OPAMP e un po' di analisi circuitale, abbiamo ottenuto che:

\begin{equation}
	\left\{
  \begin{array}{l l l}
    I_1\,=\,\frac{V_a\,-\,0}{R_1}\\
    I_2\,=\,\frac{V\ped{out}\,-\,V_a}{R_2}\\
    I_1\,=\,I_2\,+\,I_{p}^-
  \end{array} \right.\
  \label{eq:sistema}
\end{equation}

\begin{equation}
	V\ped{out}\,=\,I_2R_2\,+\,V_a
	\label{eq:offset}
\end{equation}

dove, facendo riferimento alla Figura \ref{fig:offset}, abbiamo che $I_1$ è la corrente che passa per $R_1$, $I_2$ è la corrente del ramo di retroazione negativo, $I_{p}^-$ è la corrente di polarizzazione negativa, $V_a$ è la tensione presente sul nodo dell'ingresso invertente.
Quindi con una serie di banali passaggi matemtici che sfruttano le relazioni \ref{eq:offset} e \ref{eq:sistema} otteniamo il seguente risultato:

\begin{equation}
	V\ped{out}\,=\,V_a\left(1+\frac{R_2}{R_1}\right)\,-\,I_{p}^-R_2
	\label{eq:offset1}
\end{equation}

Quindi, ricordandoci che le correnti di polarizzazione sono dell'ordine di qualche decina di $\SI{}{\nano\ampere}$, abbiamo scelto i valori delle resistenze $R_1$ e $R_2$ in modo da minimizzare il contributo di $I_{p}^-R_2$. Pertanto abbiamo scelto i seguenti valori di resistenza: $R_1\,=\,\SI{10}{\ohm}$ e $R_2\,=\,\SI{10}{\kilo\ohm}$.
Infine abbiamo ricavato il valore di $V_a$ invertendo la relazione \ref{eq:offset1} nella quale abbiamo trascurato l'effetto del termine $I_{p}^-R_2$, ottenendo:

\begin{equation}
	V_a\,\simeq\,V\ped{out}\frac{R_1+R_2}{R_1}\,=\, (0.97 \pm 0.07)\SI{}{\milli\volt}
\end{equation}

Infine abbiamo anche preso una misura diretta della tensione di offset che è risultata essere: $V_a\,=\,(1.09\pm0.01)\SI{}{\milli\volt}$. Si può osservare che i due valori sono quasi compatibili entro le loro incertezze, quindi con spirito ottimistico possiamo affermare che in entambi i casi abbiamo raggiunto dei valori plausibili.

\subsection*{Azzeramento di $V\ped{os}$ grazie al Trimmer}

In questo paragrafo ci poniamo il problema di azzerare il valore della tensione di offset $V\ped{os}$. Per farlo abbiamo realizzato il circuito riportato in Figura \ref{fig:trimmer}.
Per questo circuito, come per quelli precedenti le tensioni di alimentazione sono $V\ped{cc}^+\,=\,\SI{+15}{\volt}$ e $V\ped{cc}^-\,=\,\SI{-15}{\volt}$.
Possiamo notare che in questa configurazione il terminale non invertente è collegato al comune, mentre quello invertente è collegato direttamente al ramo di retroazione negativo, e quindi abbiamo un emitter follower.
Quindi al fine di avere una tensione di offset nulla abbiamo variato manualmente il valore della resistenza variabile $R_v$ fino a che non abbiamo ottenuto un valore nullo della tensione di output, $V\ped{out}\,=\,\SI{0}{\volt}$. In questo abbiamo effettiavamente ottenuto un OPAMP per sui collegando al comune i due ingressi la differenza di potenziale di output risulta effettivamente nulla, come per un amplificatore operazionale ideale.

Quindi ricordando che $R_v$ ha un range di valori possibili compreso tra $0$ e $10$ $\SI{}{\kilo\ohm}$ riportiamo i valori delle due resistenze $R\ped{1-4}$ e $R\ped{4-5}$ che indicano rispettivamente il valore di resistenza tra il pin 1 e 4  e tra il 4 e il 5 dell'amplificatore operazionale. i valori ottenuti sono:

\begin{equation}
	R\ped{1-4}\,=\,(6.158\,\pm\,0.001)\SI{}{\kilo\ohm} \qquad \textbf{e} \qquad R\ped{4-5}\,=\,(4.417\,\pm\,0.001)\SI{}{\kilo\ohm}
\end{equation}

Come si può otare la somma dei due valori di resistenza, che sono in serie, è maggiore di $\SI{10}{\kilo\ohm}$. questo non è errato dal momento che il valore messimo di $R_v$ e superiore a $\SI{10}{\kilo\ohm}$.

