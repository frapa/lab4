\section*{Materiale}

Il materiale utilizzato per questa esperienza di laboratorio è il seguente:

\begin{itemize} \itemsep2pt \parskip0pt \parsep0pt
    \item{Breadboard, cavetti e cavi a banana per le connessioni;}
    \item{Circuito integrato SN74LS00N a 14 pin;}
	\item{Circuito integrato SN74LS08N a 14 pin;}
	\item{Una schedina per la visualizzazione dello stato di uscite digitali a LED;}
    \item{Alimentatore di tensione continua;}
    \item{Generatore di funzioni d'onda: Agilent 33120A;}
    \item{Multimetro: Agilent Technologies 34410A;}
    \item{Oscilloscopio: Agilent DSO-X 2002A;}
\end{itemize}

Facciamo presente che sui valori di resistenza riportati in tutto l'elaborato non abbiamo riportato la loro incertezza per motivi di chiarezza e leggibilità dello scritto. Tuttavia abbiamo assunto un errore del $5\%$ sul valore nominale delle resistenze.

\begin{figure}[h]
    \centering
    \begin{subfigure}{0.44\columnwidth}
        \centering
        \begin{circuitikz}
            \draw
                (1.75, 0) node[american nand port] (nand1) {}
                (nand1.in 1) to (nand1.in 2) 
                (0, 0) node [anchor=east] {IN} -| (nand1.in 1)
                (nand1.out) node [anchor=west] {OUT}
            ;
        \end{circuitikz}
        \caption{NOT}
        \label{fig:not}
    \end{subfigure}
    \begin{subfigure}{0.54\columnwidth}
        \centering
        \begin{circuitikz}
            \draw
                (1.75, 0) node[american nand port] (nand1) {}
                (3.5, 0) node[american nand port] (nand2) {}
                (nand1.out) -| (nand2.in 1)
                (nand1.out) -| (nand2.in 2)
                (nand1.in 1) node [anchor=east] {A}
                (nand1.in 2) node [anchor=east] {B}
                (nand2.out) node [anchor=west] {OUT}
            ;
        \end{circuitikz}
        \caption{AND}
        \label{fig:and}
    \end{subfigure}
    \begin{subfigure}{\columnwidth}
        \centering
        \begin{circuitikz}
            \draw
                (1, 0) node[american nand port] (nand1) {}
                (3, 1) node[american nand port] (nand2) {}
                (3, -1) node[american nand port] (nand3) {}
                (5, 0) node[american nand port] (nand4) {}
                (nand1.out) -| (nand2.in 2)
                (nand1.out) -| (nand3.in 1)
                (nand2.out) -| (nand4.in 1)
                (nand3.out) -| (nand4.in 2)
                (nand2.in 1) to ++ (-2.5, 0) node [anchor=east] {A}
                (nand3.in 2) to ++ (-2.5, 0) node [anchor=east] {B}
                (nand1.in 1) to ++ (0, 1)
                (nand1.in 2) to ++ (0, -1)
                (nand4.out) node [anchor=west] {OUT}
            ;
        \end{circuitikz}
        \caption{XOR}
        \label{fig:xor}
    \end{subfigure}
    \caption{Tutte le altre porte logiche possono essere costruite utilizzando porte NAND.
	Per una NOT (\ref{fig:not}) è sufficiente collegare tra di loro gli ingressi,
	la AND si ottiene mettendo una NAND in serie ad un NOT, per rovesciare lo stato logico,
	mentre una XOR è un poco più complicata, ma sempre realizzabile con poche porte NAND.
	Anche la OR non è difficile da realizzare, con due not agli ingressi di una NAND.}
    \label{fig:porte}
\end{figure}
%\section*{Circuito}