\subsection{Dati e risultati}

L'esperienza è stata divisa in varie parti. Abbiamo testato il funzionamento
di alcune porte logiche nella prima parte, verificato il loro comportamento
in commutazione nella seconda, e le abbiamo poi usate per montare un paio di
circuiti nella terza parte.

\paragraph{Porte logiche.}

Le porte che abbiamo utilizzato ci sono state fornite in forma di integrati contenenti
4 porte ciascuno. In elettronica digitale è comune lavorare con porte NAND poiché con
queste è possibile costruire quasi tutti gli altri tipi di porte facilmente; in questo
modo non è necessario avere molti tipi di integrati con le varie porte. Tutte le 
porte che abbiamo usato sono del tipo TTL, ovvero operano da 0 a + 5 V.

Abbiamo quindi verificato il funzionamento delle porte NAND sull'integrato 74LS00,
grazie ad una piccola schedina che mostra con dei LED il livello (alto o basso)
delle uscite.

Altre porte che abbiamo costruito e testato sono le porte NOT, AND e XOR,
la cui costruzione utilizzando porte NAND è riportata in figura \ref{fig:porte9}

\begin{figure}
    \centering
    \footnotesize
    \begin{subfigure}{0.44\columnwidth}
        \centering
        \begin{circuitikz}
            \draw
                (1.75, 0) node[american nand port] (nand1) {}
                (nand1.in 1) to (nand1.in 2) 
                (0, 0) node [anchor=east] {IN} -| (nand1.in 1)
                (nand1.out) node [anchor=west] {OUT}
            ;
        \end{circuitikz}
        \caption{}
        \label{fig:p9}
    \end{subfigure}
    \begin{subfigure}{0.54\columnwidth}
        \centering
        \begin{circuitikz}
            \draw
                (1.75, 0) node[american nand port] (nand1) {}
                (3.5, 0) node[american nand port] (nand2) {}
                (nand1.out) -| (nand2.in 1)
                (nand1.out) -| (nand2.in 2)
                (nand1.in 1) node [anchor=east] {A}
                (nand1.in 2) node [anchor=east] {B}
                (nand2.out) node [anchor=west] {OUT}
            ;
        \end{circuitikz}
        \caption{}
        \label{fig:p9}
    \end{subfigure}
    \begin{subfigure}{\columnwidth}
        \centering
        \begin{circuitikz}
            \draw
                (1, 0) node[american nand port] (nand1) {}
                (3, 1) node[american nand port] (nand2) {}
                (3, -1) node[american nand port] (nand3) {}
                (5, 0) node[american nand port] (nand4) {}
                (nand1.out) -| (nand2.in 2)
                (nand1.out) -| (nand3.in 1)
                (nand2.out) -| (nand4.in 1)
                (nand3.out) -| (nand4.in 2)
                (nand2.in 1) to ++ (-2.5, 0) node [anchor=east] {A}
                (nand3.in 2) to ++ (-2.5, 0) node [anchor=east] {B}
                (nand1.in 1) to ++ (0, 1)
                (nand1.in 2) to ++ (0, -1)
                (nand4.out) node [anchor=west] {OUT}
            ;
        \end{circuitikz}
        \caption{}
        \label{fig:p9}
    \end{subfigure}
    \caption{}
    \label{fig:porte9}
\end{figure}
