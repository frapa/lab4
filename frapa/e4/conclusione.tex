\subsection{Conclusione}

Nonostante i circuiti si siano rivelati non proprio banali da montare (anche a causa di errori
da principianti), siamo riusciti a verificare il funzionamento di tutti e quattro i circuiti in esame.
I risultati numerici sono stati in buon accordo con la teoria, nonostante qualche deviazione significativa.

Da notare il fatto che abbiamo usato per la prima volta il feedback positivo, sia per il trigger che
per l'oscillatore. Abbiamo usato con successo due nuovi componenti (LM311 e fototransistor) in circuiti
di grande importanza applicativa.

Tutto sommato l'esperienza è stata molto interessante. Per concludere, vogliamo citare Otto Schmitt, il
geniale inventore ed ingegnere statunitense che inventò l'omonimo trigger di Schmitt imitando il sistema
nervoso dei calamari. Tra le sue alte imprese riportiamo l'aver pubblicato il suo primo articolo su Science
a 17 anni e l'aver fondato il cambo dell'ingegneria biomedica.
