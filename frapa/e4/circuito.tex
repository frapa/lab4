\subsection{Materiali e circuiti}

\subsection{Materiali e circuiti}

Per la costruzione dei quattro circuiti \ref{fig:circuits4} ci
siamo serviti dei seguenti materiali:

\begin{itemize}
    \item{Breadboard, cavi a banana e cavetti da breadboard.}
    \item{Amplificatore operazionale LM311.}
    \item{Resistenze: \SI{100}{\ohm}, 2 da \SI{10}{\kilo\ohm} e \SI{100}{\kilo\ohm}.}
    \item{Capacità da 100 nF.}
    \item{Fotodiodo}
    \item{Alimentatore di corrente continua.}
    \item{Generatore di forme d'onda Agilent 33120A.}
    %\item{Multimetro Agilent 34410A.}
    \item{Oscilloscopio Agilent DSO-X 2002A.}
\end{itemize}

\begin{figure*}[b!]
        \centering
        \small
        \begin{subfigure}[b]{0.48\textwidth}
            \def\svgwidth{\columnwidth}
            %\subimport{figure/}{comparatore.pdf_tex}
            \caption{Comparatore semplice.}
            \label{fig:comparatore4}
        \end{subfigure}
        ~
        \begin{subfigure}[b]{0.48\textwidth}
            \def\svgwidth{\columnwidth}
            %\subimport{figure/}{trigger_schmitt.pdf_tex}
            \caption{Comparatore con isteresi (trigger di Schmitt). Questo circuito
                sfrutta una retroazione negativa per evitare che il comparatore
                scatti più volte a causa del rumore (elettrico o non).}
            \label{fig:trigger_schmitt4}
        \end{subfigure}
        ~
        \begin{subfigure}[b]{0.48\textwidth}
            \def\svgwidth{\columnwidth}
            %\subimport{figure/}{oscillatore.pdf_tex}
            \caption{Oscillatore a rilassamento.}
            \label{fig:oscillatore4}
        \end{subfigure}
        ~
        \begin{subfigure}[b]{0.48\textwidth}
            \centering
            \def\svgwidth{0.8\columnwidth}
            %\subimport{figure/}{crepuscolare.pdf_tex}
            \caption{Interruttore crepuscolare.}
            \label{fig:crepuscolare4}
        \end{subfigure}

        \caption{Circuiti costruiti durante l'esperienza}
        \label{fig:circuits4}
\end{figure*}
