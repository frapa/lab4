\subsection{Dati e risultati}

Lo schema del circuito realizzato è mostrato in figura \ref{fig:circ}. Lo scopo del circuito è
quello di misurare la temperatura tramite la resistenza PT100 e di attivare o disattivare
la resistenza $R_s$ (l'elemento riscaldante) in modo che la temperatura rimanga ad un valore impostato.
È il funzionamento di base di un termostato di una casa o di un forno elettrico.
Ovviamente noi abbiamo riscaldato solo la resistenza $R_s$, per cui nel nostro circuito
$R_s$ e PT100 erano a contatto tra loro. 

Analizziamo ora il circuito con tutti i suoi blocchi funzionali, evidenziati nello schema,
in modo da capire come opera e quali accorgimenti sono necessari per farlo funzionare correttamente.
Questo circuito inizia ad essere di una certa complessità per cui è importante testare ogni blocco
funzionale separatamente. In questo modo si evita di perdere troppo tempo per capire dov'è l'errore.
Riporiamo in ogni sezione i risultati dei nostri test.

I primi tre blocchi (1, 2 e 3) servono per la lettura della temperatura, mentre il 4 comanda il circuito di riscaldamento.
Le specifiche che sono state richieste erano le seguenti: l'output dei primi 3 circuiti doveva essere una tensione
proporzionale alla temperatura, con una costante di proporzionalità di 100 mV/\si{\celsius}. Inoltre
l'output doveva essere 0 V a 0 \si{\celsius}. In questo modo la lettura della temperatura è semplicissima,
ogni grado cCelsius corrisponde a 0.1 V.

\paragraph{Generatore di tensione di riferimento.}

Per costruire il circuito di nostro interesse è importante avere a disposizione una sorgente di tensione
stabilizzata, utile in più punti del circuito. Poiché il nostro scopo era di costruire un termostato
e non uno stabilizzatore di tensione, ne abbiamo utilizzato uno integrato, il REF02. All'apparenza del tutto
simile ad un comune operazionale, questo integrato necessita di essere alimentato da una tensione da 8 V a 40 V
(nel nostro caso 15 V come gli altri operazionali presenti nel circuito) e di essere collegato a comune. L'output
dell'integrato è una tensione di 5 V con qualche millivolt di errore. Oltre ad avere una buona stabilità
ha anche un ottimo coefficiente termico, con un output che varia di soli 10 ppm/$^\circ$C.

Il REF02 in nostro possesso forniva in output 5.004 V. 

\paragraph{Conversione da resistenza a tensione.}

Il blocco 1 serve per leggere, mediante la resistenza al platino PT100 la temperatura dell'ambiente (o meglio della
resistenza $R_s$) e a trasformarla in una tensione. Questo blocco è un generatore di corrente costante da 1 mA, corrente
che passa poi attraverso la PT100 (l'ingresso dell'operazionale assorbe pochi nanoampere). Si ha quindi una tensione in uscita
proporzionale alla temperatura.
La PT100 è una resistenza che vale 100 \si{\ohm} a 0$^\circ$C e che ha un coefficiente di temperatura di 0.385 \si{\milli\ohm\per\celsius}.
La precisione nel valore della corrente è fondamentale, infatti, se per esempio al posto di 1 mA passasse 1.01 mA,
a 0$^\circ$C si avrebbe un errore di 1 \si{\ohm} nella lettura della resistenza che equivale ad un incertezza di circa \SI{3}{\celsius},
errore del tutto non trascurabile.

Poiché dal REF02 arrivano 5 V e l'ingresso invertente è un ground virtuale, per ottenere la corrente desiderata
è necessario utilizzare una resistenza da 5 \si{\kilo\ohm}. Dato che questo valore deve essere preciso, al posto
di utilizzare una resistenza da 5 \si{\kilo\ohm} con il 5\% di incertezza è consigliabile usarne una più piccola
(per esempio 4.7 \si{\kilo\ohm} come abbiamo fatto noi) e poi mettere in serie un potenziomentro $P_1$ da 1 \si{\kilo\ohm}
in modo da poter regolare a piacere la resistenza. Questo trucco è molto utile e lo abbiamo usato in altre parti del circuito.
Sempre per la stessa ragione è necessario usare un operazionale con una tensione di offset piccola.
Abbiamo utilizzato un OP07, che è costruito per avere una tensione
di offset bassa. Il produttore riporta un massimo di 75 $\mu$V, circa 20 volte di meno di un $\mu$A741. 

Per la taratura abbiamo collegato un amperometro in serie al generatore di corrente e agendo sul trimmer
ci siamo assicurati che passasse una corrente di 1 mA. Siamo riusciti ad ottenere una corrente di 1.0001 $\pm$ 1 mA!

\paragraph{Amplificatori (blocchi 2 e 3).}

L'output del blocco 1 è quindi di $100 + T\times0.385$ mV dove T è la temperatura in Celsius.
Le specifiche richieste erano di avere un output con una sensibilità di 100 mV/\si{\celsius}, 
in modo da poter leggere comodamente la temperatura e da rendere semplice la misura (per rilevare
0.385 mV serve un buon strumento). Di conseguenza, il blocco 2 è un amplificatore di tensione.
Era inoltre richiesto che a 0 \si{\celsius} l'output del blocco 3 fosse di 0 V, per poter leggere
la temperatura direttamente con il voltmetro senza dover fare calcoli. Per ottenere
questo risultato il blocco 3 è un'INA (l'AD622 come nella precedente esperienza) che amplifica
la differenza tra i suoi due ingressi. Fornendo su uno degli input il segnale amplificato proveniente
dalla resistenza al platino e all'altro una tensione opportuna è possibile eliminare la tensione
(l'offset) dovuta ai 100 \si{\ohm} a 0 \si{\celsius} della PT100.

Anche se può sembrare ovvio regolare il blocco 2 in modo che trasformi 0.385 mV in 100 mV, questa
non è la scelta più saggia, perché all'ingresso del blocco 3 si avrebbero 10 V a 0$^\circ$C e per annullare l'offset
serve quindi una tensione stabile di 10 V, mentre noi abbiamo solo 15 V e 5 V. Inoltre a temperature oltre i 35-40 \si{\celsius}
l'operazione $U_2$ rischia di andare in saturazione. Un'alternativa è la seguente:
il blocco 2 amplifica la tensione proveniente dalla PT100 in modo che a 0 \si{\celsius} il suo output sia
di 5 V. In questo modo il terzo blocco può fare il confronto tra la tensione di 5 V proveniente dal REF02 e quella
del termometro e fornire un output nullo a 0 \si{\celsius}. È inoltre necessario regolare il guadagno dell'INA $U_3$
in modo che l'uscita vari di 100 mV/\si{\celsius}.

Facciamo alcuni brevi calcoli per capire come tarare il circuito. Il blocco due deve avere un guadagno $G_2 = 50$
per trasformare i 100 mV forniti dal primo blocco a 0 \si{\celsius} in 5V, quindi due valori di resistenza opportuna
sono 100 \si{\kilo\ohm} e 2 \si{\kilo\ohm}. Il terzo blocco riceve quindi
$V_2 = 5 \;\si{\volt} + 0.385 \times G_2T\; \si{\milli\volt} = 5 + 0.01925 \times T$ V.
Vogliamo quindi che il guadagno del blocco 3 sia $G_3 = 100/19.25 = 5.195$. L'AD622 può essere tarato per avere
un guadagno $G_3$ da 1 a 1000 collegando i due piedini di guadagno con una resistenza opportuna. La formula per
calcolare questa resistenza è $R_G = \SI{50.5}{\kilo\ohm}/(G_3 - 1) \simeq \SI{12040}{\ohm}$. Per ottenere questa
resistenza abbiamo usato il solito trucco di mettere una resistenza da \SI{10}{\kilo\ohm} in serie ad un potenziometro
da \SI{10}{\kilo\ohm}, fornire in input un segnale noto e verificare che l'uscita fosse quella desiderata.

Per testare il funzionamento dei due blocchi abbiamo collegato il circuito al generatore di tensione in modo da
poter variare l'ingresso di ciascun blocco e leggere col multimetro l'uscita. In questo modo ci siamo assicurati
che i guadagni fossero impostati più precisamente possibile al valore voluto.

A questo punto siamo in possesso di un termometro di facile lettura. Abbiamo quindi testato i primi tre blocchi
collegati tra loro, misurando la temperatura del laboratorio: abbiamo letto $V_3 = 2.68$ V che vuol dire 26.8\si{\celsius}.
Confrontando il risultato con quello degli altri gruppi abbiamo trovato un buon accordo.

\paragraph{Elemento riscaldante e circuito di controllo.}

Ora che siamo in possesso di un termometro, possiamo vedere come funziona l'ultimo blocco. Quello che volgliamo
ottenere è la termostatazione di una resistenza, vale a dire mantenere una resistenza ad una determinata temperatura.
La resistenza riscaldante $R_s$ si riscalda grazie al passaggio di corrente. Deve essere collegata a +5 V e a comune
mediante un transistor di potenza, come il 2N2222 da noi utilizzato. Il transistor deve essere di potenza poiché per scaldare
la resistenza è necessario il passaggio di una grande corrente. È necessario usare una resistenza piccola, come quella
da 27 \si{\ohm} che abbiamo usato noi, per assicurare il passaggio di una grande quantità di corrente. La caduta in seturazione
da collettore a emettitore del transistor è di 0.4 V, quindi usando la resistenza riferita sopra si ottiene
$I = (V\ped{alimentazione} - V\ped{CE}\ap{sat}) / R_s = 170$ mA.

Bisogna notare che la resistenza $R_s$ non è stata alimentata dal REF02 e abbiamo allestito un canale di alimentazione e un cavo
per il comune saparato dal resto del circuito. Questo è fondamentale per il corretto funzionamento del circuito.
Il REF02 non è in grado di fornire 170 mA di corrente. Inoltre usare lo stesso filo del resto del circuito per il collegamento
a comune è molto problematico perché la corrente è ingente e questo fa si che, anche se la resistenza dei cavi è piccola,
ci siano delle differenze di potenziale tra un punto e l'altro. Avere i collegamenti a comune a diversi potenziali
all'interno del circuito è un disastro, il circuito diventa inutile.

La corrente di base $I_B$ del transistor controlla la corrente di collettore $I_C$, finchè il transitor non entra in saturazione,
con una legge approssimativamente lineare: $I_C = \beta I_B$ ($\beta \simeq 75$). Possiamo quindi agire sulla corrente di base
per accendere o spegnere la corrente che attraversa $R_s$. Questo è lo scopo del blocco 4. Inoltre poiché la legge sopra riportata
è lineare, è possibile costruire un controllo di tipo \emph{poporzionale}, ovvero un sistema che varia la corrente in base a quando
distante è la temperatura di $R_s$ da quella desiderata.

Le specifiche richieste sono le seguenti: se la temperatura è più alta di quella desiderata il transistor deve essere in interdizione,
in modo da bloccare la corrente e quindi il riscaldamento. Se la temperatura è più bassa di almeno un grado di quella desiderata,
la corrente attraverso $R_s$ deve essere massima, ovvero 170 mA. Se la temperatura è inferiore a quella desiderata ma di meno di un grado
la corrente deve essere proporzionale alla distanza dalla temperatura voluta.

La temperatura desiderata si imposta mediante il potenziometro $P_4$ con la solita scala 0.1 V/\si{\celsius} e 0 V indica 0\si{\celsius}.
Se per esempio al piedino centrale ($V_5$) di $P_4$ ci sono 4 V allora la temperatura desiderata è 40\si{\celsius}. Il potenziometro $P_4$
è stato alimentato dal REF02 e rappresenta la manopola del termostato.

Il blocco 4 è un altro amplificatore differenziale che confronta la temperatura letta e trasformata dai primi 3 blocchi con quella desiderata,
amplificando di 100 volte la differenza. Se la tensione all'input invertente $V_3$ è superiore a quella impostata significa che la temperatura
è più alta del voluto e l'amplificatore va in saturazione negativa mandando il transistor in cut-off. Se la tensione $V_3$ è minore
di quella impostata si ottiene che $V_4 = 100\times(V_5 - V_3)$. Perciò quando la differenza tra temperatura misurata e desiderata è 1\si{\celsius}
$V_4 = 10$ V. Vogliamo che a questa tensione il transistor vada in saturazione, in modo che se $V_4$ aumenta ancora la corrente rimanga fissa.
Affinché il transistor vada in saturazione è necessario che $I_C = 170 \;\si{\milli\ampere} = \beta I_B \implies I_B = 2.27$ mA.
La tensione di base $V_B$ del transistor in regione lineare è 1.3 V (a causa della caduta base-emettitore. Il valore 1.3 V è
riportato nelle specifiche del transistor), di conseguenza deve valere $R_B = (10 - 1.3) \;\si{\volt} / 2.27\;\si{\milli\ampere} = 3.8\;\si{\kilo\ohm}$.

Construito in questo modo il circuito opera perfettamente secondo le specifiche. Per testare l'ultimo blocco abbiamo semplicemente
verificato il funzionamento del circuito nel complesso, poiché le altre parti erano già state testate. Abbiamo provato ad impostare
la temperatura desiderata a 40\si{\celsius} monitornado al contempo $V_3$, $V_4$ e la corrente attraverso l'elemento riscaldante.
Abbiamo constatato che la temperatura è aumentata fino a circa 39\si{\celsius} con una corrente stabile di 170 mA e che dopo quella temperatura
la corrente è calata bruscamente, per raggiungere lo zero quando la temperatura ha superato la soglia desiderata. Dopodiché
la temperatura ha continuato ad oscillare attorno alla soglia desiderata, a volte costringendo il "controller" a mandare nuovamente il
transistor in saturazione.