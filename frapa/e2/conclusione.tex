\subsection{Conclusione}

Dopo aver calcolato le correnti di polarizzazzione, ci siamo proposti di ricalcolare
il valore di $V\ped{offset}$ con la formula \eqref{eq:law2} senza però trascurare il valore di
$I_p^-$. Abbiamo quindi inserito il valore della corrente di polarizzazione trovato nei
paragrafi precedenti non rilevando alcun cambiamento nelle cifre significative (quelle riportare)
di $V\ped{offset}$. Allo stesso modo abbiamo verificato che la tensione di offset residua
dopo l'azzeramento non variasse facendo i calcoli senza trascurare la tensione di polarizzazione.

Siamo quindi certi che le approssimazioni eseguite non inficiano la correttezza dei
risultati presentati in questa relazione. Ci riteniamo soddisfatti di tali risultati, poiché
sono in pieno accordo con i valori tipici che ci sono stati riferiti a lezione.
È stato inoltre interessante vedere quanto sia facile notare l'esistenza di queste deviazioni dall'idealità
di un amplificatore operazionale. L'esperienza è inoltre stata utile per conoscere l'ordine di grandezza di queste
deviazioni, in modo da non commettere errori (o almeno avere uno strumento in più per correggerli)
nella progettazione e realizzazione di circuiti elettronici.
