\subsection{Materiali e Circuiti}

Abbiamo realizzato i circuiti schematizzati in Figura \ref{fig:circuits}.
Ci siamo serviti dei seguenti materiali:

\begin{itemize}
    \item{Breadboard, cavi a banana e cavetti da breadboard.}
    \item{Amplificatore operazionale UA741.}
    \item{Resistenze: \SI{10}{\ohm}, \SI{10}{\kilo\ohm}, \SI{100}{\kilo\ohm}
        e una variabile per aggiustare l'offset dell'amplificatore operazionale.
        Nel nostro caso abbiamo usato una resistenza trimmer con un range operativo da 0
        a \SI{10}{\kilo\ohm}.}
    \item{Alimentatore di corrente continua.}
    \item{Multimetro Agilent 34410A.}
    \item{Oscilloscopio Agilent DSO-X 2002A.}
\end{itemize}

\begin{figure}[b]
        \onecolumn
        \centering
        \begin{subfigure}[b]{0.48\textwidth}
            \def\svgwidth{\columnwidth}
            \subimport{figure/}{v_offset.pdf_tex}
            \caption{Circuito per misurare $V\ped{offset}$. Questo circuito sfrutta l'amplificatore
                per amplificare il valore della tensione di offset e renderla più facilmente misurabile.}
            \label{fig:v_off_circ}
        \end{subfigure}
        ~
        \begin{subfigure}[b]{0.48\textwidth}
            \def\svgwidth{\columnwidth}
            \subimport{figure/}{Ip_minus.pdf_tex}
            \caption{Circuito per la misura della corrente di polarizzazione $I_p^-$. Questo circuito
                richiede la cancellazione della tensione di offset per funzionare.}
            \label{fig:ip_minus_circ}
        \end{subfigure}
        \caption{Circuiti costruiti durante l'esperienza}
        \label{fig:circuits}
\end{figure}
