\subsection{Dati e risultati}

\begin{figure}[t!]
    \def\svgwidth{0.5\textwidth}
    \subimport{figure/}{offset_graph.pdf_tex}
    \caption{La figura mostra la tensione in uscita in funzione della differenza di tensione agli input di 
        un operazionale reale e di uno ideale. La pendenza e $V\ped{offset}$ sono esagerate
        (la pendenza è molto minore di quella reale, mentre la tensione di offset è molto più grande) per
        motivi di chiarezza grafica. In un amplificatore reale, oltre
        al fatto che esiste una tensione di offset, le tensioni di saturazione non coincidono con quelle
        di alimentazione ed inoltre non sono simmetriche e neppure esattamente costanti (su un intervallo
        $V\ped{diff}$ da 0 a -15 V abbiamo misurato una variazione di 0.14 V) e il guadagno non è infinito. }
    \label{fig:v_offset_graph2}
\end{figure}

\subsubsection{Tensione di offset}

Un amplificatore operazionale ideale amplifica la differenza tra i due segnali
in ingresso. Questo significa che se i due segnali sono uguali, l'output deve essere zero.
Negli operazionali reali, questo non è vero; esiste infatti una tensione di offset
$V\ped{offset}$ tra gli ingressi per la quale l'output è nullo, e questa tensione è diversa da zero.
Questa tensione è dovuta al processo produttivo di costruzione degli operazionali.
Un amplificatore ha uno stadio di amplificazione differenziale in ingresso costruito utilizzando
transistor, che non possono mai essere prodotti in maniera perfettamente uguale. Differenti
transistor rispondono in modo anche abbastanza diverso agli input e questo causa uno sbilanciamento
negli ingressi dell'operazionale, che varia in base al tipo di transistor utilizzati.

L'esistenza della tensione di offset implica anche che un cosiddetto ground virtuale
abbia in realtà una tensione diversa da zero. In pratica si può anche vedere $V\ped{offset}$
come la differenza di potenziale tra gli ingressi.

\paragraph{Esistenza della tensione di offset.}

La figura \ref{fig:v_offset_graph2} mostra la differenza tra la situazione reale e quella
ideale. Come è ben visibile in figura, è necessario applicare una tensione di offset per
avere un output nullo. In altre parole, collegando i due input allo stesso potenziale,
nel caso ideale la tensione dovrebbe essere nulla, ma in quello reale non lo è.
Per verificare questo fatto abbiamo montato il circuito \ref{fig:v_off_exists2} e abbiamo misurato
la tensione di output. È risultato che l'output era in saturazione negativa (come in Figura \ref{fig:v_offset_graph2}),
ovvero $V\ped{out} = -12.80 \pm 0.005$ V. Collegando l'ingresso invertente con tensioni negative fino a -15 V
e vedendo che l'uscita restava circa costante (a -15 V ha raggiunto $-12.94 \pm 0.005$ V),
ci siamo accertati di essere realmente in saturazione. Abbiamo quindi verificato l'esistenza
della tensione di offset.

\paragraph{Misura della tensione di offset.}

Per misurare la tensione di offset abbiamo utilizzato il circuito \ref{fig:v_off_circ2}.
Il circuito sfrutta l'amplificatore operazionale per amplificare la tensione $V\ped{offset}$
in modo da renderla facilmente misurabile.
Poiché la differenza di potenziale tra gli ingressi è non nulla, 
$V_A \neq 0$. Si ha quindi (facendo riferimento alla figura \ref{fig:circ_con_corr2}):

\begin{equation}
    I_1  = \frac{V_A}{R_1}, \qquad I_2 = I_1 - I_p^- = \frac{V_A}{R_1} - I_p^-
\end{equation}

da cui, usando la legge di Ohm, si ottiene

\begin{equation}
    V\ped{out} = V_A + R_2 I_2 = V_A + \frac{R_2}{R_1} V_A - R_2 I_p^-
    \label{eq:law2}
\end{equation}

Questa formula riassume il funzionamento del circuito. Misurando $V\ped{out}$ è possibile
ricavare facilmente il valore di $V_A$, assumendo che il contributo dato da $I_p^-$ sia trascurabile.
Dato che la tensione di offset è la tensione che esiste tra i terminali
di input dell'operazionale (uno è messo a comune), cioè $V_A = V\ped{offset}$, possiamo così misurare $V\ped{offset}$.

Grossomodo, il funzionamento di questo circuito può essere pensato intuitivamente nel seguente modo.
Supponiamo che la tensione di offset tra non invertente ed invertente sia positiva (cioè se il non invertente è a 0 V, l'invertente è a $V\ped{offset}$ V.).
All'accensione dell'alimentazione,
$V_A = 0 V$, per cui, essendo l'amplificatore non ideale, l'uscita sarà positiva (ricordo che un amplificatore
operazionale segue una legge del tipo $V\ped{out} = G(V^+ - V^- + V\ped{offset})$, dove G è il guadagno differenziale).
Il ramo di feedback tende quindi ad alzare la d.d.p. in $V_A$ e la avvicina a $V\ped{offset}$, riducendo $V\ped{out}$.
Il ciclo di feedback si ripete (o meglio tutto si bilancia quasi istantaneamente) finché $V_A$ non diventa praticamente
indistinguibile da $V\ped{offset}$ (grazie al fatto che il guadagno di un operazionale è enorme, circa $10^5$).

\begin{figure}[t]
    \def\svgwidth{0.5\textwidth}
    \subimport{figure/}{v_offset_current.pdf_tex}
    \caption{La figura riporta il circuito \ref{fig:v_off_circ2} con indicate le correnti che scorrono nei vari rami.
		Il verso delle correnti è stato scelto coerentemente con il segno che abbiamo usato nei calcoli e per scrivere i risultati.}
    \label{fig:circ_con_corr2}
\end{figure}

Poiché, come è evidente dalla \eqref{eq:law2}, anche $I_p^-$ ha un certo rilievo, abbiamo scelto i
valori $R_1 = 10 \; \Omega$ e $R_2 = 10 \; k\Omega$, in modo da rendere il contributo del secondo termine molto
grande rispetto a quello dell'ultimo termine.

Dalle nostre misure è risultato:

\begin{equation}
    V\ped{out} = - 970 \pm 30 \;\si{\milli\volt} 
\end{equation}

da cui si calcola (abbiamo assunto 5\% di incertezza sui valori delle resistenze)

\begin{equation}
    V\ped{offset} = \frac{V\ped{out}}{1 + R_2/R_1} = - 0.97 \pm 0.07 \; \si{\milli\volt}
\end{equation}

in accordo con i valori tipici per un operazionale economico come l'UA741.

Abbiamo inoltre misurato direttamente il valore di $V_A$, perché in questo caso è stato possibile
farlo con il multimetro a nostra disposizione, ottenendo $V\ped{offset} = 1.07 \pm 0.005$ mV, risultato
questi compatibile con quello precedente.

\subsubsection{Correzione della tensione di offset}

La tensione di offset può essere di qualche millivolt, come visto nel paragrafo precedente,
e per alcune applicazioni di precisione questo può essere un grosso limite.
In tal caso è necessario usare un amplificatore operazionale con una tensione di offset minore
oppure si può annullarla utilizzando gli appositi piedini.

Gran parte degli amplificatori operazionali è infatti munita di due piedini di regolazione dell'offset
che vanno collegati agli estremi di una resistenza variabile. Il piedino centrale della resistenza
va collegato all'alimentazione. Sbilanciando la resistenza in modo la tensione alimentazione-offset
sia diversa per i due piedini, andiamo a polarizzare lo stadio differenziale in ingresso all'opamp,
aggiustando in questo modo la risposta del circuito. Riusciamo così a ridurre $V\ped{offset}$ a qualche
$\mu$\si{\volt}.

Dopo aver eseguito l'operazione, abbiamo misurato la nuova tensione di offset con lo stesso circuito
di prima (il circuito \ref{fig:v_off_circ2}). Abbiamo ottenuto

\begin{equation}
    V\ped{offset} = 3 \pm 3 \; \mu V
\end{equation}

che è notevolmente minore della precedente. Notare che in questo caso sarebbe stato molto più
difficile misurare direttamente questa differenza di potenziale, dato che è molto piccola.

\subsubsection{Correnti di polarizzazione}

Oltre alla tensione di offset, un altra deviazione dall'idealità importante degli opamp reali è l'esistenza
delle correnti di polarizzazione. Un amplificatore reale, al contrario di quello ideale, assorbe infatti una
minima quantità di corrente attraverso gli ingressi. Queste correnti variano da qualche nanoampere a qualche
femtoampere a seconda del tipo di transistor utilizzati per la costruzione dell'operazionale.

Per misurare queste correnti dobbiamo sfruttare l'amplificazione fornita dall'operazionale, poiché
le correnti sono troppo piccole per essere misurate con strumenti economici.

\paragraph{Misura della corrente di polarizzazione assorbita dall'ingresso invertente.}

Abbiamo quindi costruito il circuito \ref{fig:ip_minus_circ2}. Per la misura della corrente di polarizzazione
è fondamentale usare un trimmer per eliminare la tensione di offset prima di montare il circuito.
In questo caso vale $V_A = R_I I_p^-$. È necessario scegliere $R_I$ grande in modo da rendere $V_A$ più facilmente
misurabile. Noi abbiamo optato per una resistenza da \SI{100}{\kilo\ohm}. Il circuito funziona poi come il
precedente, vale infatti la stessa analisi circuitale del caso precedente.

In questo caso però quello che vogliamo misurare è $I_p^-$ e non $V_A$. Sostituendo in \eqref{eq:law2}
l'espressione di $V_A$, risulta:

\begin{equation}
    V\ped{out} = \left( R_I + \frac{R_2}{R_1} R_I - R_2 \right) I_p^-
\end{equation}

perciò

\begin{equation}
    I_p^- = \frac{V\ped{out}}{ R_I + \frac{R_2}{R_1} R_I - R_2}
\end{equation}

Facendo più misure di $V\ped{out}$ (l'output variava abbondantemente sul multimetro e sull'oscilloscopio),
calcolando $I_p^-$ in ogni caso e poi facendo media e deviazione standard abbiamo ottenuto

\begin{equation}
    I_p^- = 27.7 \pm 1.7 \; \si{\nano\ampere}
\end{equation}

anche questo un valore tipico per amplificatori operazionali basati su transistor BJT.

\paragraph{Misura della corrente di polarizzazione assorbita dall'ingresso non invertente.}

Una corrente di polarizzazione entrante nell'ingresso non invertente è stata scelta positiva,
cioè parliamo di corrente assorbita da tale ingresso. 

Il procedimento in questo caso è del tutto analogo, ad eccezione del circuito; in questo caso
abbiamo usato il \ref{fig:ip_plus_circ2}. La differenza sta nella posizione della resistenza $R_I$
che deve adesso trasformare $I_p^+$ in una tensione e non $I_p^-$. Poiché l'offset è stato azzerato
(o meglio, reso trascurabile, in queste misure $V_A$ è dell'ordine dei millivolt, mentre la tensione
di offset, come mostrato prima è di microvolt) $V_A = R_I I_p^+$. Vale quindi la seguente formula, ricavata
come quella del paragrafo precedente:

\begin{equation}
    I_p^+ = \frac{V\ped{out} + R_2 I_p^-}{ R_I + \frac{R_2}{R_1} R_I}
\end{equation}

Con lo stesso procedimento di prima abbiamo quindi ottenuto

\begin{equation}
    I_p^+ = -29 \pm 3 \; \si{\nano\ampere}
\end{equation}
