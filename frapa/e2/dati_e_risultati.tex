\subsection{Dati e risultati}

\paragraph{Tensione di offset}

Un amplificatore operazionale ideale amplifica la differenza tra i due segnali
in ingresso. Questo significa che se i due segnali sono uguali, l'output deve essere zero.
Negli operazionali reali, questo non è vero; esiste infatti una tensione di offset
$V\ped{offset}$ tra gli ingressi per la quale l'output è nullo, e questa tensione è diversa da zero.
Questa tensione è dovuta al processo produttivo di costruzione degli operazionali.
Un amplificatore ha uno stadio di amplificazione differenziale in ingresso costruito utilizzando
transistor, che non possono mai essere prodotti in maniera perfettamente uguale. Differenti
transistor rispondono in modo anche abbastanza diverso agli input e questo causa uno sbilanciamento
negli ingressi dell'operazionale, che varia in base al tipo di transistor utilizzati.

\begin{figure}[H]
    \def\svgwidth{0.5\textwidth}
    \subimport{figure/}{offset_graph.pdf_tex}
    \caption{La figura mostra la tensione in uscita in funzione della differenza di tensione agli input di 
        un operazionale reale e di uno ideale. La pendenza e $V\ped{offset}$ sono esagerate
        (la pendenza è molto minore di quella reale, mentre la tensione di offset è molto più grande) per
        motivi di chiarezza grafica. In un amplificatore reale, oltre
        al fatto che esiste una tensione di offset, le tensioni di saturazione non coincidono con quelle
        di alimentazione ed inoltre non sono simmetriche e neppure esattamente costanti (su un intervallo
        $V\ped{diff}$ da 0 a -15 V abbiamo misurato una variazione di 0.14 V) e il guadagno non è infinito. }
    \label{fig:v_offset_graph}
\end{figure}

\emph{Esistenza della tensione di offset.}
La Figura \ref{fig:v_offset_graph} mostra la differenza tra la situazione reale e quella
ideale. Come è ben visibile in figura, è necessario applicare una tensione di offset per
avere un output nullo. In altre parole, collegando i due input allo stesso potenziale,
nel caso ideale la tensione dovrebbe essere nulla, ma in quello reale non lo è.
Per verificare questo fatto abbiamo montato il circuito \ref{fig:v_off_exists} e abbiamo misurato
la tensione di output. È risultato che l'output era in saturazione negativa (come in Figura \ref{fig:v_offset_graph}),
ovvero $V\ped{out} = -12.80 \pm 0.005$ V. Collegando l'ingresso invertente con tensioni negative fino a -15 V
e vedendo che l'uscita restava circa costante (a -15 V ha raggiunto $-12.94 \pm 0.005$ V),
ci siamo accertati di essere realmente in saturazione. Abbiamo quindi verificato l'esistenza
della tensione di offset.

\emph{Misura della tensione di offset.}
Per misurare la tensione di offset abbiamo utilizzato il circuito \ref{fig:v_off_circ}.

\paragraph{Correzione della tensione di offset}

\paragraph{Correnti di polarizzazione}
