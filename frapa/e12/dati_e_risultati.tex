\subsection{Dati e risultati}

\begin{figure*}[t]
    \centering
    \begin{circuitikz}[x=0.9cm, y=0.9cm]
        \foreach \i in {1,...,4} {
            \draw (\i*3.8, 0) -- (\i*3.8, 3) -- (\i*3.8 + 2, 3) -- (\i*3.8 + 2, 0) -- (\i*3.8, 0);
            \draw (\i*3.8, 2.25) node[right] (J\i) {J\i};
            \draw (\i*3.8, 0.75) node[right] (K\i) {K\i};
            \draw (\i*3.8 + 2, 2.25) node[left] (Q\i) {Q\i};
            \draw (\i*3.8 + 2, 0.75) node[left] {$\overline{\text{Q\i}}$};
            \draw (\i*3.8 + 1, 3) node[anchor=north] (PR\i) {PR};
            \draw (\i*3.8 + 1, 0) node[anchor=south] (CL\i) {CL};
            \draw (\i*3.8, 1.25) -- (\i*3.8 + 0.25, 1.5) -- (\i*3.8, 1.75);
            \coordinate (CLK\i) at (\i*3.8, 1.5);
            \coordinate (preCLK\i) at (\i*3.8 - 0.5, 1.5);
            \coordinate (preK\i) at (\i*3.8 - 0.2, 0.75);
        };
        \coordinate (init) at (1, -1);
        \draw
            (1, -2) node[rground] {}
            to [C, l=470 nF] (init) node[left] {A}
        ;
        \draw
            (init) to [R, l=10<\kilo\ohm>] (1, 1)
            node[anchor=south] {$V\ped{CC}$}
        ;
        \draw
            (init) -| (2.5, 4) -| (PR1)
            (init) -| (CL2)
            (init) -| (CL3)
            (init) -| (CL4)
            
            (Q1) -| (J2.west)
            (Q1) --++ (1, 0) --++ (0, 4.5) node[above] {Q1}
            (Q1) --++ (1, 0) |- (preK2) ++ (0.1, 0) circle (0.1cm)
            
            (Q2) -| (J3.west)
            (Q2) --++ (1, 0) --++ (0, 4.5) node[above] {Q2}
            (Q2) --++ (1, 0) |- (preK3) ++ (0.1, 0) circle (0.1cm)
            
            (Q3) -| (J4.west)
            (Q3) --++ (1, 0) --++ (0, 4.5) node[above] {Q3}
            (Q3) --++ (1, 0) |- (preK4) ++ (0.1, 0) circle (0.1cm)
            
            (Q4.east) --++ (0.5, 0) |- (3, 6) |- (J1.west)
            (Q4.east) --++ (1, 0) --++ (0, 4.5) node[above] {Q4}
            (3, 6) |- (preK1) ++ (0.1, 0) circle (0.1cm)
        ;
        \draw
            (1, 4.75) node[left] (CLK) {CLK}
            (CLK) -| (preCLK1) -- (CLK1)
            (CLK) -| (preCLK2) -- (CLK2)
            (CLK) -| (preCLK3) -- (CLK3)
            (CLK) -| (preCLK4) -- (CLK4)
        ;
        \draw
            (CL1.south) ++ (0, -0.1) circle (0.1) ++ (0, -0.1) --++ (0, -0.5) node[right] {$V\ped{CC}$}
            (PR2.north) ++ (0, 0.1) circle (0.1) ++ (0, 0.1) --++ (0, 0.75) node[above] {$V\ped{CC}$}
            (PR3.north) ++ (0, 0.1) circle (0.1) ++ (0, 0.1) --++ (0, 0.75) node[above] {$V\ped{CC}$}
            (PR4.north) ++ (0, 0.1) circle (0.1) ++ (0, 0.1) --++ (0, 0.75) node[above] {$V\ped{CC}$}
        ;
    \end{circuitikz}
    \caption{Shift resister ciclico.}
    \label{fig:shift11}
\end{figure*}

\paragraph{Shift register.}

Gli shift register, o registri a scorrimento, sono circuiti che mantengono in memoria un certo numero di bit
e spostano la sequenza di 1 bit ogni volta che arriva un fronte d'onda negativo (I FF da noi usati erano negative edge triggered).
I registri a scorrimento
servono principalmente per costruire convertitori seriale-parallelo e parallelo-seriale, ma trovano
impiego anche in altri campi.


La figura \ref{fig:shift11} mostra lo shift register che abbiamo costruito. Il registro
è costruito con 4 flip-flop JK e di conseguenza memorizza 4 bit. Inoltre è l'uscita dell'ultimo FF
è collegata con l'entrata del primo, per fare un registro ciclico, che cicla la sequenza di bit.

Vediamo come funziona il circuito. La prima cosa da fare all'accensione del circuito è caricare
una sequenza di bit in memoria. Questo risultato è ottenuto grazie alle porte preser (PR) e clear (CL),
che, come fa intuire il nome, servono rispettivamente per importare un valore all'uscita Q o a
``pulire'' i valori precedenti, impostando 0. Entrambe queste entrate sono negate, cioè vengono impostate
a 1 se non si vuole usarle. Infatti le porte preset dei FF 2, 3 e 4 e la clear del primo sono
state collegate tutte a $V\ped{CC}$, in modo che non siano mai utilizzate. Il trucco viene poi con
la resistenza e il condensatore; il punto A è collegato con il preset del primo FF
e ai clear dei restanti. All'accensione dell'alimentazione il condensatore impiega un certo
tempo per caricarsi, per cui all'inizio il punto A è a 0 logico e carica un 1 sul primo FF,
mentre gli altri sono impostati a 0 mediante le porte clear.

Ogni volta che arriva un fronte d'onda negativo (cioè c'è un passaggio da 1 a 0) nel clock,
i flip-flop leggono i loro ingressi e copiano il valore all'uscita. Poiché ci sono
dei ritardi di pochi nanosecondi tra l'arrivo del fronte e la copiatura del valore in ingresso
si ha che la sequenza di bit viene spostata a destra (si noti che l'ultimo valore $Q_4$ viene spostato
a $Q_1$).

Abbiamo quindi collegato il circuito alla schedina con i LED che ci permette di visualizzare i valori
0 o 1 alle uscite $Q_1,\dots,Q_4$. Fornendo un onda quadra di frequenza bassa (1-10 Hz) è stato possibile
vedere il bit 1 impostato all'inizio ciclare.

\paragraph{Contatore a 8 bit.}

Un utilissima applicazione dei FF sono i contatori. A parte il loro utilizzo in moltissime componenti
elettroniche, in ambito scientifico sperimentale sono molto utili per realizzare esperimenti
in cui sia necessario contare degli oggetti, per esempio il numero di particelle arrivate su di un rivelatore.
In queste applicazioni si usano degli integrati come il 74LS191, che includono al loro interno tutto il necessario
per il conteggio. Non resta che collegarli opportunamente.

Tuttavia i 74LS191 sono contatori a 4 bit e noi vogliamo realizzare un contatore a 8 bit, per
cui ce ne servono 2 in cascata, il secondo dei quali conta il numero delle volte che il primo
è andato in overflow. Il circuito in figura \ref{fig:contatore11} è il circuito adatto allo scopo.

Gli integrati 74LS191 hanno le seguenti porte:

\begin{itemize}
    \item{CE (Count Enable): se bassa il contatore è abilitato, altrimenti il contatore non fa nulla.}
    \item{D/$\overline{\text{U}}$ (Down/Up): se 0 il contatore conta da 0 (0000) a 15 (1111), mentre se
        è alta il conteggio avviene al rovescio.}
    \item{RC (Ripple Clock): quest'uscita è normalmente a 1, ma scende a 0 quando si è raggiunto il valore massimo
        o minimo, in base alla direzione in cui si sta contando (crescente o decrescente rispettivamente).
        Rimane bassa finché il clock non ritorna basso (il 74LS191 è positive edge triggered).}
    \item{LOAD e $P_1,\dots,P_4$: Quando LOAD è basso il segnale agli ingressi $P_1,\dots,P_4$ viene copiato
        in modo asincrono alle uscite, indipendentemente dal clock. Utile per impostare i valori iniziali.}
\end{itemize}

La parte in basso a sinistra del circuito è composta da un interruttore che permette di scegliere la direzione
del conteggio. Il FF JK serve per assicurarsi che lo stato delle porte U/$\overline{\text{U}}$ venga cambiato
soltanto quando il clock è alto, poiché sulle specifiche del 74LS191 è scritto che cambiarlo in altri momenti
causa uno stato non definito all'interno dell'integrato. 

\begin{figure*}
    \centering
    \begin{circuitikz}[x=0.9cm, y=0.9cm]
        \draw
            (3, -4) -| (5, -7) -| (3, -4)
            (3, -4.75) node[right] (J) {J}
            (3, -6.25) node[right] (K) {K}
            (5, -4.75) node[left] (Q) {Q}
            (5, -6.25) node[left] {$\overline{\text{Q}}$}
            (3, -5.25) -- (3.25, -5.5) -- (3, -5.75)
            (4, -4) node[below] (PR) {PR} to[short, o-o] ++ (0, 0.5) node[above] {$V\ped{CC}$}
            (4, -7) node[above] (CL) {CL} to[short, o-o] ++ (0, -0.5) node[below] {$V\ped{CC}$}
        ;
        \coordinate (preCLK_JK) at (2, -6);
        \coordinate (preJ) at (2.5, -4.75);
        \coordinate (preK) at (2.5, -6.25);
        \coordinate (CLK_JK) at (3, -6);
        
        \draw
            (-1, -6) node[above] {$V\ped{CC}$} |- (0, -7) ++ (0.1, 0) circle (0.1)
            ++ (0.1, 0) --++ (0.5, 0.3) ++ (0, -0.3) circle (0.1) ++ (0.1, 0)
            coordinate (switch)
        ;
        
        \draw
            (switch) -| (preJ) -- (J.west)
            (switch) -| (preK) to [short, -o] (K.west)
            (switch) --++ (0.5, 0) to[R, l=1<\kilo\ohm>] ++(0, -1.5) node[rground] {}
        ;
    
        \draw
            (-1, 0) node[left] (CLK) {CLK} -| (preCLK_JK) -- (CLK_JK)
        ;
        
        \foreach \i in {0,1} {
            \draw (11, -\i*6 + 2) -| (14, -\i*6 - 3) -| (11, -\i*6 + 2);
            \draw
                (11, -\i*6 + 2 - 0.5) node[right] (P0\i) {P0}
                (11, -\i*6 + 2 - 1) node[right] (P1\i) {P1}
                (11, -\i*6 + 2 - 1.5) node[right] (P2\i) {P2}
                (11, -\i*6 + 2 - 2) node[right] (P3\i) {P3}
                (11, -\i*6 + 2 - 3.5) node[right] (CE\i) {CE}
                (11, -\i*6 + 2 - 4) node[right] (DU\i) {D/$\overline{\text{U}}$}
                (11, -\i*6 + 2 - 4.5) node[right] (LOAD\i) {LOAD}
                (11, -\i*6 + 2 - 2.5) --++ (0.25, -0.25) --++ (-0.25, -0.25)
                (14, -\i*6 + 2 - 0.5) node[left] (Q0\i) {Q0}
                (14, -\i*6 + 2 - 1) node[left] (Q1\i) {Q1}
                (14, -\i*6 + 2 - 1.5) node[left] (Q2\i) {Q2}
                (14, -\i*6 + 2 - 2) node[left] (Q3\i) {Q3}
                (14, -\i*6 + 2 - 3.5) node[left] (CR\i) {RC}
            ;
            \draw
                (P0\i) --++ (-1, 0)
                (P1\i) --++ (-1, 0)
                (P2\i) --++ (-1, 0)
                (P3\i) --++ (-1, 0) |- ++ (-1, 2) node[rground] {}
                (Q0\i.east) to[short, -o]++ (1, 0)
                (Q1\i.east) to[short, -o]++ (1, 0)
                (Q2\i.east) to[short, -o]++ (1, 0)
                (Q3\i.east) to[short, -o]++ (1, 0)
            ;
            \coordinate (CLK\i) at (11, -\i*6 + 2 - 2.75);
            \coordinate (preCE\i) at (10.5, -\i*6 + 2 - 3.5);
        };

        \draw
            (Q) --++ (2, 0) |- (DU0.west)
            (Q) ++ (2, 0) |- (DU1.west)
            (CLK) --++ (7, 0) |- (CLK0)
            (CLK) ++ (7, 0) |- (CLK1)
            (CR0.east)  to[short, o-] ++ (1, 0) |- ++ (-6.5, -2) |- (preCE1) to[short, -o] (CE1.west)
            (CE0.west) to[short, o-] ++ (-1, 0) node[rground] {}
        ;
        \draw
            (LOAD0.west) to[short, o-] ++(-0.1, 0) -| (7.5, -8.5)
            (LOAD1.west) to[short, o-] ++(-0.1, 0) -| (7.5, -11)
            --++ (1, 0)
        ;
        \coordinate (LS) at (8.5, -11);
        \draw
            (LS) to[C, l=470 nF] ++ (0, -1) node[rground] {}
            (LS) to[R, l_=10<\kilo\ohm>] ++(0, 1.5) node[above] {$V\ped{CC}$}
        ;
    \end{circuitikz}
    \caption{Contatore a 8 bit.}
    \label{fig:contatore11}
\end{figure*}
