\subsection{Materiali e circuiti}

\begin{itemize}
    \item{Breadboard, cavi a banana e cavetti da breadboard.}
    \item{Amplificatore operazionale LM311 e amplificatore strumentale AD622.}
    \item{Resistenze: 4 da \SI{10}{\ohm}, 2 da \SI{100}{\ohm}, 2 da \SI{100}{\kilo\ohm}, 5 da \SI{10}{\kilo\ohm} e
        un trimmer da \SI{10}{\kilo\ohm}.}
    \item{2 diodi.}
    \item{Resistenza al platino PT100.}
    \item{Alimentatore di corrente continua.}
    \item{Generatore di forme d'onda Agilent 33120A.}
    \item{Multimetro Agilent 34410A.}
    \item{Oscilloscopio Agilent DSO-X 2002A.}
\end{itemize}

\begin{figure*}[b!]
        \centering
        \small
        \begin{subfigure}[b]{0.43\textwidth}
            \def\svgwidth{\columnwidth}
            \subimport{figure/}{raddrizzatore.pdf_tex}
            \caption{Semplice raddrizzatore a mezz'onda costruito con un UA741.}
            \label{fig:raddrizzatore5}
        \end{subfigure}
        \quad
        \begin{subfigure}[b]{0.53\textwidth}
            \def\svgwidth{\columnwidth}
            \subimport{figure/}{radd_ott.pdf_tex}
            \caption{Raddrizzatore migliorato.}
            \label{fig:rad_ott5}
        \end{subfigure}
        ~
        \begin{subfigure}[b]{0.48\textwidth}
            \def\svgwidth{\columnwidth}
            \subimport{figure/}{diff.pdf_tex}
            \caption{Amplificatore differenziale usato per filtrare il rumore.}
            \label{fig:diff5}
        \end{subfigure}
        ~
        \begin{subfigure}[b]{0.48\textwidth}
            \def\svgwidth{\columnwidth}
            \subimport{figure/}{inst_amp.pdf_tex}
            \caption{Circuito di verifica del funzionamento dell'amplificatore strumentale AD622.}
            \label{fig:inst_amp5}
        \end{subfigure}

        \caption{Circuiti costruiti durante l'esperienza}
        \label{fig:circuits5}
\end{figure*}
