\subsection{Dati e risultati}

\paragraph{Flip-Flop SR}

Il flip-flop SR (Set-Reset, il nome deriva dalle terminazioni) è il primo elemento
di memoria che incontriamo in questo corso. È un circuito sequenziale e non
combinatorio, nel senso che il suo ouput dipende da quello che è avvenuto prima
e non dallo stato delle entrate.

\begin{figure}
	\centering
	\begin{circuitikz}
		\draw
			(2, 0) node[nand port] (s) {}
			(2, 3) node[nand port] (r) {}
			(5, 0) node[nand port] (q) {}
			(5, 3) node[nand port] (qbar) {}
			(0, 0) node[left] {S} -| (s.in 1) -| (s.in 2)
			(0, 3) node[left] {R} -| (r.in 1) -| (r.in 2)
			(s.out) -| (q.in 2)
			(r.out) -| (qbar.in 1)
			(q.out) --++ (0.5, 0) node[right] {Q} --++ (0, 0.5) 
			(qbar.out) --++ (0.5, 0) node[right] {$\bar{\text{Q}}$} --++ (0, -0.5)
			(qbar.in 2) --++ (0, -0.5) -- (5.65, 0.5)
			(q.in 1) --++(0, 0.5) -- (5.65, 2.5)
		;
	\end{circuitikz}
	\caption{}
	\label{fig:ff_sr11}
\end{figure}

Il flip-flop SR è riportato in figura \ref{fig:ff_sr11}. Questo piccolo e straordinario circuito
si basa sulla dipendenza reciproca tra entrate ed uscite delle porte NAND.

\paragraph{Flip-Flop SR sincronizzarto}

\begin{figure}
	\centering
	\begin{circuitikz}
		\draw
			(2, 0) node[nand port] (s) {}
			(2, 3) node[nand port] (r) {}
			(5, 0) node[nand port] (q) {}
			(5, 3) node[nand port] (qbar) {}
			(0, 0) node[left] {S} -| (s.in 2)
			(0, 3) node[left] {R} -| (r.in 1)
			(s.out) -| (q.in 2)
			(r.out) -| (qbar.in 1)
			(q.out) --++ (0.5, 0) node[right] {Q} --++ (0, 0.5) 
			(qbar.out) --++ (0.5, 0) node[right] {$\bar{\text{Q}}$} --++ (0, -0.5)
			(qbar.in 2) --++ (0, -0.5) -- (5.65, 0.5)
			(q.in 1) --++(0, 0.5) -- (5.65, 2.5)
			(0, 1.5) node[left] (clk) {CLK} -| (r.in 2)
			(clk) -| (s.in 1)
		;
	\end{circuitikz}
	\caption{}
	\label{fig:ff_sr_sync11}
\end{figure}

\paragraph{Flip-Flop RS D-type}

\begin{figure}
	\centering
	\begin{circuitikz}[transform shape, scale=0.95]
		\draw
			(2, 0) node[nand port] (s) {}
			(2, 3) node[nand port] (r) {}
			(5, 0) node[nand port] (q) {}
			(5, 3) node[nand port] (qbar) {}
			(-0.75, 3) node[not port] (not) {}
			(-2, 0) node[left] (d) {D} -| (s.in 2)
			(d) -| (not.in)
			(not.out) -- (r.in 1)
			(s.out) -| (q.in 2)
			(r.out) -| (qbar.in 1)
			(q.out) --++ (0.5, 0) node[right] {Q} --++ (0, 0.5) 
			(qbar.out) --++ (0.5, 0) node[right] {$\bar{\text{Q}}$} --++ (0, -0.5)
			(qbar.in 2) --++ (0, -0.5) -- (5.65, 0.5)
			(q.in 1) --++(0, 0.5) -- (5.65, 2.5)
			(0, 1.5) node[left] (clk) {CLK} -| (r.in 2)
			(clk) -| (s.in 1)
		;
	\end{circuitikz}
	\caption{}
	\label{fig:ff_sr_dtype11}
\end{figure}

\paragraph{Antirimbalzo}

\begin{figure}
	\centering
	\begin{circuitikz}
		\draw
			(0, 1.5) node[rground] (gnd) {}
			(1, 1.5) node[spdt] (i) {}
			(gnd) -- (i.in)
			(5, 0) node[nand port] (q) {}
			(5, 3) node[nand port] (qbar) {}
			(i.out 2) --++ (0, -1.46) -| (q.in 2)
			(i.out 1) --++ (0, 1.46) -| (qbar.in 1)
			(q.out) --++ (0.5, 0) node[right] {Q} --++ (0, 0.5) 
			(qbar.out) --++ (0.5, 0) node[right] {$\bar{\text{Q}}$} --++ (0, -0.5)
			(qbar.in 2) --++ (0, -0.5) -- (5.65, 0.5)
			(q.in 1) --++(0, 0.5) -- (5.65, 2.5)
			(qbar.in 1) ++ (-1, 0) to[R, l=1 k] ++(0, 1.75) node[anchor=south] {$V\ped{cc}$}
			(q.in 2) ++ (-1, 0) to[R, l=1 k] ++(0, -1.75) node[anchor=north] {$V\ped{cc}$}
		;
	\end{circuitikz}
	\caption{}
	\label{fig:antirimbalzo11}
\end{figure}

\paragraph{Divisore di frequenza}

\begin{figure}
	\centering
	\begin{circuitikz}
		\draw
		
		;
	\end{circuitikz}
	\caption{}
	\label{fig:freq_div11}
\end{figure}
