\subsection{Materiali e circuiti}

Ci siamo serviti dei seguenti materiali:

\begin{itemize}
    \item{Breadboard, cavi a banana e cavetti da breadboard.}
    \item{Amplificatore operazionale UA741.}
    \item{Resistenze: \SI{100}{\ohm}, \SI{2.2}{\kilo\ohm}, \SI{10}{\kilo\ohm}, \SI{100}{\kilo\ohm}
        e una variabile per aggiustare l'offset dell'amplificatore operazionale.
        Nel nostro caso abbiamo usato una resistenza trimmer con un range operativo da 0
        a \SI{10}{\kilo\ohm}.}
    \item{Condensatore da 180 pF.}
    \item{Alimentatore di corrente continua.}
    \item{Generatore di forme d'onda Agilent 33120A.}
    \item{Multimetro Agilent 34410A.}
    \item{Oscilloscopio Agilent DSO-X 2002A.}
\end{itemize}

Per le misure abbiamo utilizzato i circuiti riportati in figura \ref{fig:circuits3},
che verranno spiegati nei seguenti paragrafi.

\begin{figure*}[t]
        \centering
        \small
        \begin{subfigure}[b]{0.32\textwidth}
            \def\svgwidth{\columnwidth}
            \subimport{figure/}{slew_circ.pdf_tex}
            \caption{Circuito utilizzato per misuraro lo slew rate dell'opamp. Il circuito è
		riportato sul manuale dell'operazionale ed è lo standard per questo tipo di misure.}
            \label{fig:slew_circ3}
        \end{subfigure}
        ~
        \begin{subfigure}[b]{0.32\textwidth}
            \def\svgwidth{\columnwidth}
            \subimport{figure/}{max_curr_circ.pdf_tex}
            \caption{Per la misura della massima corrente erogabile dall'operazionale ci siamo
		serviti del circuito in figura.}
            \label{fig:max_curr_circ3}
        \end{subfigure}
        ~
        \begin{subfigure}[b]{0.32\textwidth}
            \def\svgwidth{\columnwidth}
            \subimport{figure/}{banda_circ.pdf_tex}
            \caption{Circuito di cui abbiamo misurato la banda passante. Il circuito è stato utilizzato
		in due differenti configurazioni: una con guadagno di 20 dB e $R_2 = \SI{10}{\kilo\ohm}$
		e l'altro con un guadagno di 20 dB e $R_2 = \SI{100}{\kilo\ohm}$.}
            \label{fig:banda_circ3}
        \end{subfigure}
        ~
        \begin{subfigure}[b]{0.32\textwidth}
            \def\svgwidth{\columnwidth}
	    \vspace{0.5cm}
            \subimport{figure/}{guad_freq_bassa_circ.pdf_tex}
	    \vspace{0pt}
            \caption{Circuito utilizzato per misurare il guadagno open loop di un amplificatore
		operazionale. Il circuito è pensato per fare misure a basse frequenze, dove
		l'operazionale ha un guadagno molto elevato (~ $10^5$). Quando l'operazionale viene
		usato ad alte frequenze il suo guadagno si riduce notevolmente e il circuito in figura
		diventa inutilizzabile. In questi regimi si utilizza il circuito \ref{eq:gain_high_circ3}.}
            \label{fig:gain_low_circ3}
        \end{subfigure}
        ~
        \begin{subfigure}[b]{0.32\textwidth}
            \def\svgwidth{\columnwidth}
            \subimport{figure/}{guad_freq_alta_circ.pdf_tex}
	    \vspace{0pt}
            \caption{Circuito per la misura del guadagno open loop ad alte frequenze. In circuito in pratica
		serve a fare una misura diretta del guadagno, tuttavia utilizza un ramo di feedback per impedire
		saturazioni non volute (per esempio causate dalla tensione di offset, che non può mai essere
		perfettamente oppure da altre tensioni DC dovute all'alimentazione che vengono amplificate moltissimo
		perché a bassa frequenza).}
            \label{fig:gain_high_circ3}
        \end{subfigure}

        \caption{Circuiti costruiti durante l'esperienza}
        \label{fig:circuits3}
\end{figure*}
